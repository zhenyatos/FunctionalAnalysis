\documentclass[../main.tex]{subfiles}

\begin{document}

\subsection{[Простые функции]}

Пусть $(X, \mathcal{M}, \mu)$ -- измеримое пространство. Функция $f : X \to \R$ называется простой, если она имеет вид:
\begin{equation}\label{eq:1:1:4:1}
	f(t) = \sum_{i=1}^{n}{c_i \Ind_{E_i} (t)}
\end{equation}
Где $E_i \in \mathcal{M}$, а $\Ind_{E_i}$ -- индикатор множества $E_i$. Отметим, что у одной и той же функции может быть несколько представлений вида \eqref{eq:1:1:4:1}, поэтому проверять эквивалентность функций в том или ином смысле (как равенство функций, как равенство интегралов Лебега\footnote{см. далее} и т.д.) -- затруднительно. Для решения этой проблемы, введем понятие о каноническом представлении простой функции. 

\begin{definition}
	Для простой функции $f$ представление вида \eqref{eq:1:1:4:1} называется каноническим, если:
	\begin{enumerate}
		\item $c_i \neq 0$
		\item $c_1 < ... < c_n$
		\item $E_i \cap E_j = \varnothing, \forall i \neq j$
	\end{enumerate}
\end{definition}

Довольно очевидно, что такое представление единственно. Менее тривиален вопрос существования. Его мы сейчас и исследуем. 

Начнем с дизъюнктности, т.е. с удовлетворения 3-му пункту канонического представления.
\begin{lemma}
	Для всякой простой функции $f : X \to \R$ существует представление вида:
	\begin{equation*}
		f(t) = \sum_{i=1}^{m}{a_i A_i (t)}
	\end{equation*}
	Где $A_i \neq \varnothing$ и $A_i \cap A_j = \varnothing, \forall i \neq j$.
\end{lemma}
\begin{proof}
	Пусть $f$ -- простая функция, тогда она имеет вид \eqref{eq:1:1:4:1}. Введем обозначения:
	\begin{align*}
		E &= \bigcup_{i=1}^{n}{E_i} \\
		E_i^c &= E \setminus E_i
	\end{align*}
	
	Если предстоящее построение испугает, то можно посмотреть на пример после него и все станет понятно. Итак, ведем следующие множества, индексируемые кортежами $u$ из $\{0, 1\}^n$:
	\begin{equation}\label{eq:1:1:4:2}
		A_u = \bigcap_{i=1}^{n}{B_i^{u_i}}
	\end{equation}
	Где:
	$$B_i^{u_i} = \begin{cases}
		E_{u_i}, &u_i = 0 \\
		E_{u_i}^c, &u_i = 1
	\end{cases}$$
	Заметим, что $A_u \cap A_v = \varnothing$ при $u, v \in \{0, 1\}^n$ т.ч. $u \neq v$. Кроме того:
	$$A_{(1, ..., 1)} = \bigcap_{i=1}^{n}{E_i^c} = \bigcap_{i=1}^{n}{(E \setminus E_i)} = E \setminus \bigcup_{i=1}^{n}{E_i} = E \setminus E = \varnothing$$
	Т.е. множество, соответствующее наибольшему индексу $u$ (при стандартном порядке на декартовом произведении) -- всегда пусто. Его и остальные пустые множества, мы, в дальнейшем, исключим из рассмотрения. Из аксиом $\sigma$-кольца следует, что множества $A_u$ измеримы т.е. $A_u \in \mathcal{M}$
	
	\begin{example}[внутри док-ва]
		Пусть $f(t) = c_1 \Ind_{E_1} (t) + c_2 \Ind_{E_2} (t)$, тогда:
		\begin{align*}
			A_{(0, 0)} &= E_1 \cap E_2 \\
			A_{(1, 0)} &= ((E_1 \cup E_2) \setminus E_1) \cap E_2 = E_2 \setminus E_1 \\
			A_{(0, 1)} &= E_1 \cap ((E_1 \cup E_2 \setminus E_2)) = E_1 \setminus E_2 \\
			A_{(1, 1)} &= \varnothing
		\end{align*}
		Теперь можно нарисовать диаграмму Эйлера и понять в чем логика построения, как мы получили из потенциально пересекающихся множеств -- непересекающиеся.
	\end{example}

	Пусть $I = \{u \in \{0, 1\}^n \ | \ A_u \neq \varnothing\}$. Выделяем индексы, соответствующие непустым множествам и далее, для них определяем $a_u$ следующим образом. Возьмем произвольное $t \in A_u, u \in I$, тогда $t \in E_i$ для всех $i$ т.ч. $u_i = 0$. Значит:
	\begin{equation}\label{eq:1:1:4:3}
		a_u = f(t) = \sum_{i \in \{1, ..., n\} : u_i = 0}{c_i}
	\end{equation}

	$I$ -- конечное множество, значит его элементы можно перенумеровать. В итоге, из \eqref{eq:1:1:4:2} и \eqref{eq:1:1:4:3}, получаем искомое представление.
\end{proof}

Теперь убедимся, что мы можем удовлетворить 1-му и 2-му пунктам канонического представления, сохранив 3-ий пункт. Основная идея -- склейка множеств $E_i$ с одинаковым значением $c_i$, а как это реализовать с помощью операций теории множеств, смотрим в доказательстве.

\begin{lemma}
	Пусть $f : X \to \R$ -- простая функция, имеющая вид:
	\begin{equation*}
		f(t) = \sum_{i=1}^{m}{c_i \Ind_{E_i} (t)}
	\end{equation*}
	Где $E_i \neq \varnothing$ и $E_i \cap E_j = \varnothing, \forall i \neq j$. Тогда найдутся такие измеримые множества $\{E_i^\prime\}_{i=1}^{n}$ и соответствующие им числа $\{c_i^\prime\}_{i=1}^{n}$, что: $c_i^\prime \neq 0$, \\ $c_1^\prime < ... < c_n^\prime$ и
	\begin{equation}\label{eq:1:1:4:4}
		f(t) = \sum_{i=1}^{n}{c_i^\prime \Ind_{E_i^\prime} (t)}
	\end{equation}
\end{lemma}
\begin{proof}
	Пусть $I = \{1, ..., m\}$. Введем вспомогательную функцию \\ $\varphi : I \to \R$ вида:
	$$\varphi(i) = c_i$$
	С её помощью мы можем выделить подмножества индексов, с одинаковыми значениями $c_i = c$, а именно, взяв прообраз $\varphi^{-1} (c)$. Сразу выделим $I_0 = \varphi^{-1} (0)$ т.к. мы хотим избавиться от множеств $E_i$ т.ч. $c_i = 0$. Оставшееся множество индексов $I \setminus I_0$ разбивается на $n \leq m$ классов эквивалентности:
	$$\{I_1, ..., I_n\} = \{\varphi^{-1} (c_i)\}_{i \in I \setminus I_0}$$
	Понятно, что возможно $\varphi^{-1} (c_i) = \varphi^{-1} (c_j)$ и при $i \neq j$, поэтому $n \leq m$. 
	
	Отметим, что по построению, классы $I_1, ..., I_n$ таковы, что:
	$$\forall i, j \in I_k : \varphi(i) = \varphi(j)$$
	Но при $k_1 \neq k_2$:
	$$\forall i \in I_{k_1} \ \forall j \in I_{k_2} : \varphi(i) \neq \varphi(j)$$
	Поэтому, можно положить $c_k^\prime = \varphi(i) = c_i, i \in I_k$ и понятно, что $c_{k_1} \neq c_{k_2}$ при $k_1 \neq k_2$.
	Соответствующие множества строим через объединения дизъюнктных множеств $E_i$:
	\begin{equation*}
		E_k^\prime = \bigcup_{i \in I_k}{E_i}
	\end{equation*}

	Набор множеств $\{E_1^\prime, ..., E_n^\prime\}$ и соответствующие им числа $\{c_1^\prime, ..., c_n^\prime\}$ уже удовлетворяют большей части условий в выводе теоремы. Элементарно упорядочить по возрастанию $c_i^\prime$, поэтому, без потери общности, будем считать, что они упорядочены. 
	
	Проверка \eqref{eq:1:1:4:4} теперь очевидна.
\end{proof}

Итак, с помощью двух сформулированных и доказанных лемм, любую простую функцию можно привести к каноническому виду. Теперь определим интеграл Лебега для простой функции.

\begin{definition}
	Пусть $f$ -- простая функция, рассмотрим её каноническое представление:
	\begin{equation*}
		f(t) = \sum_{i=1}^{n}{c_i \Ind_{E_i} (t)}
	\end{equation*}
	Тогда, интеграл Лебега от функции $f$ мы определим следующим образом:
	\begin{equation}\label{eq:1:1:4:5}
		\int_{X}{f (t) d\mu (t)} := \sum_{i=1}^{n}{c_i \mu (E_i)} 
	\end{equation} 
	При условии, что указанная сумма задана корректно (не возникла неопределенность $+\infty - \infty$). Если, кроме того, она конечна (т.е. $< +\infty$), тогда простая функция называется интегрируемой.
\end{definition}

Приведем пример вычисления интеграла Лебега для простой функции.
\begin{example}
	Пусть простая функция $f: \R^2 \to \R$ задана как:
	$$f = 1 \cdot \Ind_{[0, 1) \times [0, 1)} + 2 \cdot \Ind_{[0, 1) \times [0, 2)}$$
	Сначала приведем её к каноническому виду:
	$$f = 2 \cdot \Ind_{[0, 1) \times [1, 2)} + 3 \cdot \Ind_{[0, 1) \times [0, 1)}$$ 
	Тогда, интеграл Лебега:
	$$\int_{\R^2}{f(t) d\mu(t)} = 2 \cdot (1 - 0) \cdot (2 - 1) + 3 \cdot (1 - 0) \cdot (1 - 0) = 5$$
\end{example}

Еще более простой, но поучительный пример.
\begin{example}
	Пусть простая функция $f : \R \to \R$ задана как:
	$$f = 1 \cdot \Ind_{[0, +\infty)}$$
	Она уже в каноническом виде и:
	$$\int_{\R}{f(t) d\mu(t)} = 1 \cdot \mu([0, +\infty)) = +\infty$$
	Хотя формально (!) мы и можем вычислить интеграл, но важно то, что функцию $f$ мы не считаем интегрируемой.
\end{example}

Заметим, что в сумме \eqref{eq:1:1:4:5} порядок слагаемых не имеет значения. Таким образом, из канонического представления, принципиальное значение для корректного вычисления интеграла Лебега, имеет лишь дизъюнктность множеств $E_i$ в представлении функции. 

\end{document}