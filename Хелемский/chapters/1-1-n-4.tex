\documentclass[../main.tex]{subfiles}

\begin{document}

\subsection{[Простые функции]}

Пусть $(X, \mathcal{M}, \mu)$ -- измеримое пространство. Функция $f : X \to \R$ называется простой, если она имеет вид:
\begin{equation}\label{eq:1:1:4:1}
	f(t) = \sum_{i=1}^{n}{c_i \Ind_{E_i} (t)}
\end{equation}
Где $\mu(E_i) < \infty$, а $\Ind_{E_i}$ -- индикатор множества $E_i$. Отметим, что у одной и той же функции может быть несколько представлений вида \eqref{eq:1:1:4:1}, поэтому проверять эквивалентность функций в том или ином смысле (как равенство функций, как равенство интегралов Лебега\footnote{см. далее} и т.д.) -- затруднительно. Для решения этой проблемы, введем понятие о каноническом представлении простой функции. 

\begin{definition}
	Для простой функции $f$ представление вида \eqref{eq:1:1:4:1} называется каноническим, если:
	\begin{enumerate}
		\item $c_i \neq 0$
		\item $c_1 < ... < c_n$
		\item $E_i \cap E_j = \varnothing, \forall i \neq j$
	\end{enumerate}
\end{definition}

Довольно очевидно, что такое представление единственно. Менее тривиален вопрос существования. Его мы сейчас и исследуем. 

Начнем с дизъюнктности, т.е. с удовлетворения 3-му пункту канонического представления.
\begin{lemma}
	Для всякой простой функции $f : X \to \R$ существует представление вида:
	\begin{equation*}
		f(t) = \sum_{i=1}^{m}{a_i A_i}
	\end{equation*}
	Где $\mu(A_i) < \infty, A_i \neq \varnothing$ и $A_i \cap A_j = \varnothing, \forall i \neq j$.
\end{lemma}
\begin{proof}
	Пусть $f$ -- простая функция, тогда она имеет вид \eqref{eq:1:1:4:1}. Введем обозначения:
	\begin{align*}
		E &= \bigcup_{i=1}^{n}{E_i} \\
		E_i^c &= E \setminus E_i
	\end{align*}
	
	Если предстоящее построение испугает, то можно посмотреть на пример после него и все станет понятно. Итак, ведем следующие множества, индексируемые кортежами $u$ из $\{0, 1\}^n$:
	\begin{equation}\label{eq:1:1:4:2}
		A_u = \bigcap_{i=1}^{n}{B_i^{u_i}}
	\end{equation}
	Где:
	$$B_i^{u_i} = \begin{cases}
		E_{u_i}, &u_i = 0 \\
		E_{u_i}^c, &u_i = 1
	\end{cases}$$
	Заметим, что $A_u \cap A_v = \varnothing$ при $u, v \in \{0, 1\}^n$ т.ч. $u \neq v$. Кроме того:
	$$A_{(1, ..., 1)} = \bigcap_{i=1}^{n}{E_i^c} = \bigcap_{i=1}^{n}{(E \setminus E_i)} = E \setminus \bigcup_{i=1}^{n}{E_i} = E \setminus E = \varnothing$$
	Т.е. множество, соответствующее наибольшему индексу $u$ (при стандартном порядке на декартовом произведении) -- всегда пусто. Его и остальные пустые множества, мы, в дальнейшем, исключим из рассмотрения. Элементарно доказывается, что множества $A_u$ измеримы (из аксиом $\sigma$-кольца) и имеют конечную меру (из монотонности).
	
	\begin{example}[внутри док-ва]
		Пусть $f(t) = c_1 \Ind_{E_1} (t) + c_2 \Ind_{E_2} (t)$, тогда:
		\begin{align*}
			A_{(0, 0)} &= E_1 \cap E_2 \\
			A_{(1, 0)} &= ((E_1 \cup E_2) \setminus E_1) \cap E_2 = E_2 \setminus E_1 \\
			A_{(0, 1)} &= E_1 \cap ((E_1 \cup E_2 \setminus E_2)) = E_1 \setminus E_2 \\
			A_{(1, 1)} &= \varnothing
		\end{align*}
		Теперь можно нарисовать диаграмму Эйлера и понять в чем логика построения, как мы получили из потенциально пересекающихся множеств -- непересекающиеся.
	\end{example}

	Пусть $I = \{u \in \{0, 1\}^n \ | \ A_u \neq \varnothing\}$. Выделяем индексы, соответствующие непустым множествам и далее, для них определяем $a_u$ следующим образом. Возьмем произвольное $t \in A_u, u \in I$, тогда $t \in E_i$ для всех $i$ т.ч. $u_i = 0$. Значит:
	\begin{equation}\label{eq:1:1:4:3}
		a_u = f(t) = \sum_{i \in \{1, ..., n\} : u_i = 0}{c_i}
	\end{equation}

	$I$ -- конечное множество, значит его элементы можно перенумеровать. В итоге, из \eqref{eq:1:1:4:2} и \eqref{eq:1:1:4:3}, получаем искомое представление.
	
\end{proof}

\end{document}