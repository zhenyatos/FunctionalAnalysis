\documentclass[../main.tex]{subfiles}

\begin{document}

\subsection{Пространства $l_\infty (X)$}

Пусть $X$ -- произвольное множество, тогда определим $l_\infty (X)$ как пространство всех ограниченных функций из $\Cn^X$ и снабдим его т.н. sup-нормой:

\begin{equation}
	\norm{x}_\infty = \sup_{t \in X}{|x(t)|}
\end{equation}

Введем обозначение $l_\infty = l_\infty (\N)$ для пространства ограниченных последовательностей. Заметим, что пространства последовательностей вложены друг в друга:
\begin{equation*}
	c_{00} \subset c_0 \subset c \subset l_\infty (\N)
\end{equation*}

Посмотрим, какие еще пространства можно ввести, взяв за основу $l_\infty (X)$.

\begin{example}
	Пусть $\Omega$ -- топологическое пространство. Тогда в $l_\infty (\Omega)$ можно выделить подпространство $C_b (\Omega)$ ограниченных непрерывных функций. 
	
	Заметим, что непрерывность вовсе не обязательно влечет ограниченность, поэтому нижний индекс $b$ (от англ. <<boundary>>) важен. К примеру, если взять $\Omega = (0, 1)$ со стандартной топологией, то функция $x(t) = \frac{1}{t}$ является непрерывной, но не является ограниченной, поэтому $x \notin C_b (\Omega)$.
	
	Теперь покажем, что это подпространство замкнуто. Пусть $\{x_n\}_{n\in\N} \subset C_b (\Omega)$ т.ч. $x_n \to x$ для некоторого $x \in l^\infty (\Omega)$. Это означает, что:
	\begin{equation*}
		\norm{x_n - x} \to 0, n \to \infty
	\end{equation*}
	Фиксируем $\varepsilon > 0$, тогда $\exists N \in \N: \norm{x_n - x}_\infty < \frac{\varepsilon}{4}, \forall n \geq N$. Выбор $\frac{\varepsilon}{4}$ для оценки вместо $\varepsilon$ будет понятен из дальнейших выкладок. 
	
	Пусть $t_0 \in \Omega$, тогда $\forall t \in \Omega$:
	\begin{equation*}
		|x(t) - x(t_0)| \leq |x(t) - x_N (t)| + |x_N (t) - x_N (t_0)| + |x_N(t_0) - x(t_0)|
	\end{equation*}
	Отсюда:
	\begin{equation}\label{eq:1:1:2:1}
		|x(t) - x(t_0)| \leq 2\norm{x_N - x}_\infty + |x_N (t) - x_N (t_0)|
	\end{equation}
	Поскольку $x_N$ -- непрерывна, то $\exists U_{t_0} \subset \Omega$ -- окрестность точки $t_0$ т.ч. 
	$$|x_N (t) - x_N (t_0)| < \frac{\varepsilon}{2}, \forall t \in U_{t_0}$$
	Теперь, из \eqref{eq:1:1:2:1}, с учетом оценки на $\norm{x_N - x}$, получаем:
	\begin{equation*}
		|x(t) - x(t_0)| < 2 \cdot \frac{\varepsilon}{4} + \frac{\varepsilon}{2} = \varepsilon, \forall t \in U_{t_0}
	\end{equation*}
	В силу произвольности $\varepsilon > 0$ мы доказали, что $x$ -- непрерывна в $t_0$. В силу произвольности $t_0 \in \Omega$, доказана непрерывность $x$, т.е., что $x \in C_b (\Omega)$, что и требовалось.
\end{example}

\begin{example}
	$C[a, b] = C_b ([a, b])$ -- пространство непрерывных функций на отрезке $[a, b]$ числовой прямой. Ограниченность опускается, ведь, как известно, функция непрерывная на компакте является ограниченной.
\end{example}

\end{document}