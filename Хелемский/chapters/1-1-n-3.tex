\documentclass[../main.tex]{subfiles}

\begin{document}

\subsection{[Измеримые пространства]}

Хелемский определяет измеримые пространства следующим образом. Пусть:
\begin{enumerate}
	\item $X$ -- некоторое множество
	\item $\mathcal{M} \subset 2^X$ -- $\sigma$-кольцо его подмножеств
	\item $\mu : \mathcal{M} \to [0, +\infty]$ -- $\sigma$-аддитивная функция, т.н. мера\footnote{соглашение: $+\infty + a = +\infty, \forall a \in [0, +\infty]$}
\end{enumerate}
Тогда $(X, \mathcal{M}, \mu)$ -- измеримое пространство. 

\begin{example}
	Пусть $X = \N$, $\mathcal{M} = 2^\N$. Определим функцию $\bullet : \mathcal{M} \to [0, +\infty]$ следующим образом:
	\begin{equation}
		\bullet (A) = \begin{cases}|A|, &A - \text{конечное} \\ +\infty, &A - \text{бесконечное}\end{cases}
	\end{equation}
	Покажем, что $\bullet$ -- мера. Пусть $\{A_n\}_{n \in \N} \subset \mathcal{M}$ и $A \in \mathcal{M}$, при этом \\ $A_i \cap A_j = \varnothing, i \neq j$ и:
	$$A = \bigcup_{i=1}^{\infty}{A_i}$$
	Пусть $I = \{i \in \N \ | \ A_i \neq \varnothing \}$. Возможно два случая:
	
	\begin{enumerate}
		\item $I$ -- счетно, тогда $A \supset \bigcup\limits_{i \in I}{A_i}$ -- бесконечное и $|A_i| \geq 1, \forall i \in I$, поэтому:
		$$\sum_{i=1}^{\infty}{\bullet(A_i)} \geq \sum_{i \in I}{1} = +\infty$$
		Т.е.:
		$$\bullet(A) = +\infty = \sum_{i=1}^{\infty}{\bullet(A_i)}$$
		\item $I$ -- конечно, тогда $A$ -- конечное как объединение конечного числа конечных множеств и т.к. $|A_j| = |\varnothing| = 0, \forall j \in \N \setminus I$, имеем:
		$$\bullet(A) = |A| = \sum\limits_{i \in I}{|A_i|} = \sum\limits_{i=1}^{\infty}{\bullet(A_i)}$$
	\end{enumerate}
	
	Итак, $\sigma$--аддитивность доказана. $(\N, 2^\N, \bullet)$ -- измеримое пространство и $\bullet$ известна как считающая мера.
\end{example}

Почему Хелемский выбрал $\sigma$-кольца, а не $\sigma$-алгебры в определении измеримого пространства? Во-первых, дело в том, что для определения меры на $\mathcal{M}$ имеет значение лишь следующее\footnote{хотя можно определить меру и на полукольцах, но это чуть более заморочно}:
\begin{equation*}
	A_1, ..., A_n, ... \in \mathcal{M} \Rightarrow \bigcup_{i=1}^{\infty}{A_i} \in \mathcal{M}
\end{equation*}
Во-вторых, можно построить разумный пример измеримого пространства, такого что $X \notin \mathcal{M}$ т.е. $\mathcal{M}$ -- не $\sigma$-алгебра.

\begin{example}
	Пусть $X = \R$, $\mu^* : 2^\R \to [0, +\infty]$ -- стандартная внешняя мера (Лебега) и:
	$$\mathcal{M} = \{A \subset 2^\R \ | \ \mu^* (A) = 0\}$$
	Понятно, что $\mathcal{M}$ -- $\sigma$-кольцо, в силу полуаддитивности $\mu^*$. Пусть $E \in \mathcal{M}$ -- единица нашего $\sigma$-кольца, тогда, поскольку $\{x\} \in \mathcal{M}, \forall x \in \R$, по определению единицы имеем:
	$$\{x\} \subset E \Leftrightarrow, \forall x \in \R$$
	Т.е.:
	$$x \in E, \forall x \in \R$$
	И $E = \R$, но $\mu^* (\R) = +\infty \neq 0$ т.е. $E \notin \mathcal{M}$, противоречие. Значит, в $\mathcal{M}$ нет единицы и это не $\sigma$-алгебра.
	
	Нетрудно видеть, что $\mu^*$ -- $\sigma$-аддитивна на множествах из $\mathcal{M}$ т.к.:
	\begin{equation*}
		\mu^* \left(\bigcup_{i=1}^{\infty}{A_i}\right) = 0 = \sum_{i=1}^{\infty}{\mu^* (A_i)}, \forall \{A_i\}_{i \in \N} \subset \mathcal{M}
	\end{equation*}
	Итого, $(\R, \mathcal{M}, \mu^* |_{\mathcal{M}})$ -- измеримое пространство множеств нулевой меры на вещественной прямой. 
\end{example}


\end{document}