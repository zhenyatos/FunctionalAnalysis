\documentclass[12pt,a4paper]{article}
\usepackage[utf8]{inputenc}
\usepackage[english,russian]{babel}
\usepackage{hyperref}
\usepackage{amsthm,amssymb,amsfonts,amsmath}
\usepackage{xcolor}
\usepackage[left=3cm,right=3cm,
top=3cm,bottom=3cm,bindingoffset=0cm]{geometry}
\usepackage{thmtools}

\theoremstyle{definition}
\newtheorem{theorem}{Теорема}
\newtheorem{definition}{Определение}[section]
\newtheorem{corollarydf}{Замечание}[definition]
\newtheorem{corollaryth}{Замечание}[theorem]
\newtheorem{corollary}{Следствие}[theorem]
\newtheorem{proposition}{Утверждение}[section]

\newcommand{\Real}{\mathbb{R}}
\newcommand{\Cmplx}{\mathbb{C}}
\newcommand{\Natural}{\mathbb{N}}
\newcommand{\norm}[1]{\left\lVert#1\right\rVert}
\newcommand{\setbuild}[2]{\{#1\:|\:#2\}}
\DeclareMathOperator{\Lin}{Lin}
\DeclareMathOperator{\dist}{dist}
\DeclareMathOperator*\lowlim{\underline{lim}}
\DeclareMathOperator{\sign}{sign}
\newcommand{\tick}[1]{#1^{\prime}}
\newcommand{\dtick}[1]{#1^{\prime\prime}}
\newcommand{\bounded}[2]{\textrm{B}(#1, #2)}
\newcommand{\linear}[2]{\textrm{L}(#1, #2)}
\newcommand{\seq}[2]{\{#1\}_{#2}}
\newcommand{\seqbound}[3]{\{#1\}_{#2}^{#3}}
\newcommand{\conj}[1]{\left(#1\right)^*}
\newcommand{\weakto}{\rightharpoonup}

\hypersetup{
	colorlinks   = true, 
	urlcolor     = blue, 
	linkcolor    = blue, 
	citecolor   = red
}

%opening
\title{Линейные функционалы}
\date{}

\begin{document}

\maketitle

\begin{abstract}
	Конспект лекций по данному разделу, читаемых Сергеем Валерьевичем Лупуляком в СПбПУ. Видеозаписи:
	\begin{enumerate}
		\item \url{https://www.youtube.com/watch?v=NyO3tRUWof0}
		\item \url{https://www.youtube.com/watch?v=FkBqjrC1pjI}
		\item \url{https://www.youtube.com/watch?v=zlR53CFNsS8}
		\item \url{https://www.youtube.com/watch?v=ZxErqaR8mHU}
		\item \url{https://www.youtube.com/watch?v=Oeaqzm3VUxs}
		\item \url{https://www.youtube.com/watch?v=y6rOmdRSggE}
	\end{enumerate}
	
\end{abstract}

\newpage

\section{Теорема Хана-Банаха}

% Определение сопряженного пространства
\begin{definition}
	Пусть $X$ - н.п. над $\Real(\Cmplx)$, пространством \textcolor{red}{сопряженным к $X$} называем $X^*=\bounded{X}{\Real}(\bounded{X}{\Cmplx})$. Причем $f\in X^*\Rightarrow D(f)=X$
\end{definition}
\begin{corollarydf}
	Ключевая в этом параграфе теорема даст своего рода обоснование указанному в определении равенству $X^*=\bounded{X}{\Real}$, хотя более строго следовало бы написать $X^*=\setbuild{f\in \bounded{X}{\Real}}{D(f)=X}$ т.к. область определения ограниченного линейного функционала вообще говоря может быть произвольным линейным многообразием
\end{corollarydf}
\begin{corollarydf}
	Как известно если $Y$ - б.п., то $\bounded{X}{Y}$ - б.п. относительно нормы оператора $$\norm{f}_{X^*}=\sup_{\norm{x}\leq1}|f(x)|$$, а значит каким бы не было н.п. $X$, $X^*$ всегда б.п. (т.к. $\Real,\Cmplx$ - б.п.)
\end{corollarydf}

% Теорема Хана-Банаха
\begin{theorem}[\textbf{Хана-Банаха}]\label{th:1}
	Пусть 
	\begin{enumerate}
		\item $X$ - н.п.
		\item $L\subset X$ - линейное многообразие
		\item $f\in \bounded{X}{\Real}, D(f) = L$
	\end{enumerate}
	тогда $\exists F\in X^*$ такой что
	\begin{enumerate}
		\item $F$ - продолжение $f$
		\item $\norm{F}_{X^*} = \norm{f}_{\bounded{X}{\Real}}$
	\end{enumerate}
\end{theorem}
\begin{proof}
	Докажем только для частного случая когда $X$ - сепарабельно. \newline \newline
	
	\textbf{1.} Пусть $x_0 \notin L$ (если такого не существует, то $L=X$ и утверждение тривиально) и пусть $P_{x_0} = \Lin{\{x_0\}} = 
	\setbuild{x\in X}{\exists\alpha\in\Real: x=\alpha x_0}$ (прямая натянутая на $x_0$). Пусть $L_0 = L + P_{x_0}$. \newline
	\newline
	
	\textbf{2.} Докажем что $L_0 = L\oplus P_{x_0}$. \newline
	Пусть $0=y_0 + tx_0$, где $y_0\in L$. Если $t\neq0$, то $x_0 = -\frac{y_0}{t} \in L$ - противоречие. Значит $t=0$ и соответственно $y_0 = 0$. Представление нуля единственно $0 = 0 + 0$, откуда теперь замечаем что если для некоторого $x\in L_0$: $x = y_1 + t_1 x_0 = y_2 + t_2 x_0$, то $0 = (y_1 - y_2) + (t_1 - t_2) x_0$ и очевидно $y_1 - y_2 \in L$, $(t_1 - t_2)x_0 \in P_{x_0}$ так что $y_1 = y_2 \wedge t_1 = t_2$. \newline
	\newpage
	
	\textbf{3.} Будем производить некоторые преобразования и оценки. Хоть и сложно сходу понять зачем, но увидим. Пусть $\tick{x}, \dtick{x} \in L$, тогда \newline\newline
	$f(\tick{x}) - f(\dtick{x}) = 
	f(\tick{x} - \dtick{x}) \leq |f(\tick{x} - \dtick{x})| \leq
	\norm{f}\norm{\tick{x} - \dtick{x}} =
	\norm{f}\norm{\tick{x} \pm x_0 - \dtick{x}} \leq \\
	\leq \norm{f}\norm{\tick{x} + x_0} + \norm{f}\norm{x_0 + \dtick{x}}$ \newline
	Итого 
	$$f(\tick{x}) - \norm{f}\norm{\tick{x} + x_0} \leq  f(\dtick{x}) + \norm{f}\norm{\dtick{x} + x_0}, \forall \tick{x},\dtick{x} \in L$$ 
	откуда
	$$\exists C \in \Real: \sup_{y \in L}{(f(y) - \norm{f}\norm{y + x_0})} \leq C \leq
	\inf_{y \in L}{(f(y) + \norm{f}\norm{y + x_0})}$$ \newline
	для последнего перехода важно было что в левой части только $\tick{x}$, а в правой только $\dtick{x}$, наконец замечаем
	$$f(y) - \norm{f}\norm{y + x_0} \leq C \leq f(y) + \norm{f}\norm{y + x_0}, \forall y \in L$$
	что равносильно
	\begin{equation}\label{eq:1}
		|f(y) - C| \leq \norm{f}\norm{y + x_0}, \forall y \in L 
	\end{equation}
	\newline
	
	\textbf{4.} Пусть $x \in L_0$, тогда из \textbf{2} $\exists!{y \in L, t \in \Real}: x = y + tx_0$, соответственно можно задать $$F(x) = f(y) - tC$$, покажем что $F \in \linear{L_0, \Real}$. Действительно для $x_1, x_2 \in L_0$ имеем \newline $x_1 = y_1 + t_1 x_0, x_2 = y_2 + t_2 x_0$, где $y_1, y_2 \in L$ и $t_1, t_2 \in \Real$, тогда
	\begin{align*}
		\alpha_1 x_1 + \alpha_2 x_2 &= \alpha_1 (y_1 + t_1 x_0) + \alpha_2 (y_2 + t_2 x_0), \forall \alpha_1, \alpha_2 \in \Real  \\
		&\Leftrightarrow \\
		\alpha_1 x_1 + \alpha_2 x_2 &= (\alpha_1 y_1 + \alpha_2 y_2) + (\alpha_1 t_1 + \alpha_2 t_2)x_0, \forall \alpha_1, \alpha_2 \in \Real
	\end{align*}
	 Заметим что в $\alpha_1 y_1 + \alpha_2 y_2 \in L$ т.к. $L$ - линейное многообразие \newline и $\alpha_1 t_1 + \alpha_2 t_2 \in \Real$ так что с учетом того что $L_0 = L \oplus P_{x_0}$ и из определения $F$ получаем
	\begin{align*}
		F(\alpha_1 x_1 + \alpha_2 x_2) &= f(\alpha_1 y_1 + \alpha_2 y_2) - (\alpha_1 t_1 + \alpha_2 t_2)C, \forall \alpha_1, \alpha_2 \in \Real \\ &\Leftrightarrow \\    
		F(\alpha_1 x_1 + \alpha_2 x_2) &= \alpha_1 f(y_1) + \alpha_2 f(y_2) - \alpha_1 t_1 C - \alpha_2 t_2 C, \forall \alpha_1, \alpha_2 \in \Real \\ &\Leftrightarrow \\
		F(\alpha_1 x_1 + \alpha_2 x_2) &= \alpha_1 (f(y_1) - t_1 C) + \alpha_2 (f(y_2) - t_2 C), \forall \alpha_1, \alpha_2 \in \Real \\ &\Leftrightarrow \\
		F(\alpha_1 x_1 + \alpha_2 x_2) &= \alpha_1 F(x_1) + \alpha_2 F(x_2), \forall \alpha_1, \alpha_2 \in \Real
	\end{align*}
	Пусть $x\in L$, тогда $x = x + 0x_0 \in L_0$ и $F(x) = f(x) - 0C = f(x)$, так что $F$ - действительно продолжение $f$. Далее пусть $x \notin L$, тогда $t \neq 0$ и 
	$$|F(x)| = |f(y) - tC| = |t||f(\frac{y}{t}) - C| \overset{~\eqref{eq:1}}{\leq} |t|\norm{f}\norm{\frac{y}{t} + x_0} = \norm{f}\norm{y + tx_0} = \norm{f}\norm{x}$$, так что вообще $\forall x \in L_0: \norm{F(x)} \leq \norm{f}\norm{x}$. Мы доказали что $F \in \bounded{L_0}{\Real}$, причем $\norm{F} \leq \norm{f} \Rightarrow \norm{F} = \norm{f}$ т.к. норма продолжения $f$ всегда не меньше нормы $f$.
	\newline
	
	\textbf{5.} По условию $X$ - сепарабельно т.е. $\exists \seq{x_k}{k \in \Natural} \subset X: X = \overline{\seq{x_k}{k \in \Natural}}$. \newline
	
	\noindent Пусть $x_1 \notin L, L_1 = L \oplus P_{x_1}$ и воспользовавшись \textbf{1-4} построим $f_1$ - продолжение $f$ на $L_1$ ($f_1 \in \bounded{L_1}{\Real}, \norm{f_1} = \norm{f}$). \newline
	
	\noindent Пусть $x_2 \notin L_1, L_2 = L_1 \oplus P_{x_2}$ и аналогично построим $f_2$ - продолжение $f_1$ на $L_2$. \newline
	
	\noindent Процесс продолжения приводит к последовательности вложенных линейных многообразий $L \subset L_1 \subset L_2 \subset ... \subset L_k \subset ...$. Если начиная с некоторого $K \in \Natural$ имеем $L_k = X, \forall k \geq K$, то теорема доказана. В противном случае пусть $$X_0 = \bigcup_{k=1}^{\infty}{L_k}$$, тогда заметим что $\seq{x_k}{k \in \Natural} \subset X_0$ и соответственно $\overline{X_0} = X$. \newline
	
	\noindent $x \in X_0 \Rightarrow \exists k \in \Natural: x \in L_k$, пусть $F(x) = f_k (x)$ и видим что $$|F(x)| = |f_k (x)| \leq \norm{f_k}\norm{x} = \norm{f}\norm{x}$$ т.е. $F \in \bounded{X_0}{\Real}$, а значит можно воспользоваться теоремой о продолжении ограниченного линейного отображения со всюду плотной областью определения и теорема доказана. 
\end{proof}
\begin{corollaryth}
	Возвращаясь к теории линейных операторов - ограниченный оператор можно было продлить со всюду-плотного линейного многообразия, а ограниченный функционал можно с произвольного. Однако, уже не единственным образом.
\end{corollaryth}

% Следствие 1
\begin{corollary}\label{corll:1}
	Пусть $x_0 \neq 0$, тогда \\ $\exists f\in X^*: (\norm{f}_{X^*} = 1) \wedge (f(x_0) = \norm{x_0})$
\end{corollary}
\begin{proof}
	Пусть $P_{x_0} = \setbuild{x\in X}{\exists t\in\Real: x = tx_0}$ и пусть $x\in P_{x_0}$, тогда $x=tx_0$ для некоторого $t$ и положим $f(x)=t\norm{x_0}$. Получили $f:P_{x_0} \to \Real$, теперь докажем что $f(x)\in \linear{P_{x_0}}{\Real}$ \\
	
	Действительно если $x_1,x_2 \in P_{x_0}$, то $\exists t_1,t_2\in\Real: (x_1 = t_1 x_0) \wedge (x_2 = t_2 x_0)$, а значит $$\alpha_1 x_1 + \alpha_2 x_2 = \alpha_1 t_1 x_0 + \alpha_2 t_2 x_0 = (\alpha_1 t_1 + \alpha_2 t_2)  x_0 \in P_{x_0}$$, и $$f(\alpha_1 x_1 + \alpha_2 x_2) = (\alpha_1 t_1 + \alpha_2 t_2)\norm{x_0} = \alpha_1 (t_1 \norm{x_0}) + \alpha_2 (t_2 \norm{x_0}) = \alpha_2 f(x_1) + \alpha_2 f(x_2)$$ \\
	
	Теперь заметим что $$|f(x)| = |t|\norm{x_0} = \norm{tx_0} = \norm{x}, \forall x \in P_{x_0}$$ таким образом мы доказали что $f\in\bounded{X}{\Real}$, $D(f) = P_{x_0}$ и $\norm{f} = 1$, а также как не трудно видеть $x_0 = 1\cdot x_0$ т.е. $f(x_0) = \norm{x_0}$ по определению. Все необходимые нам свойства выполнены, но этот функционал задан только на одномерном линейном многообразии. Вот теперь мы и применяем теорему Хана-Банаха, продляя его на все пространство и мы получили искомый $f\in X^*$
\end{proof}

% Следствие 2
\begin{corollary}\label{corll:2}
	Пусть $x\in X$ и $\forall f \in X^* : f(x)=0$, тогда $x = 0$
\end{corollary}
\begin{proof}
	Внимательно смотрим на предыдущее следствие. Если $x \neq 0$, то $\exists f \in X^*: f(x) = \norm{x} \neq 0$ что противоречит условию.
\end{proof}

% Следствие 3
\begin{corollary}\label{corll:4}
	Пусть $L\subset X$ - лин. многообразие, а $x\in X: \dist(x,L) = d > 0$, тогда $\exists f\in X^*$:
	\begin{enumerate}
		\item $f(x)=1$
		\item $f(y)=0,\forall y \in L$
		\item $\norm{f}_{X^*} = \frac{1}{d}$
	\end{enumerate}
\end{corollary}
\begin{proof}
	Пусть $L_0 = L \oplus P_{x}$, напомним что по определению прямой суммы $\forall z \in L_0 \exists!y\in L, t\in\Real: z = y + tx$. Определим функционал $f : L_0 \to \Real$ так: $f(z)=t$. Из определения $f$ сразу получаем что $$f(x)=1 \text{ т.к. } x = 0 + 1\cdot x \in L_0$$, а также $$f(y) = 0, \forall y \in L \text{ т.к. } y = y + 0\cdot x \in L_0, \forall y \in L$$ т.е. мы доказали 1. и 2. \\
	
	Должно быть очевидно (после двух доказательств такого вида) что он линеен т.е. $f\in\linear{L_0}{\Real}$. Пусть $z \in L_0 \setminus L$, тогда $z \neq 0$ т.к. $0\in L$ и 
	$$|f(z)| = |t| = 
	|t|\frac{\norm{z}}{\norm{z}} = 
	\frac{|t|\norm{z}}{\norm{y+tx}} =
	\frac{\norm{z}}{\norm{\frac{y}{t} + x}} = 
	\frac{\norm{z}}{\norm{x - (-\frac{y}{t})}}$$
	, т.к. $-\frac{y}{t} \in L$ и $\dist(x, L) = d > 0$, имеем $\norm{x - (-\frac{y}{t})} \geq d$, откуда
	$$|f(z)| = \frac{\norm{z}}{\norm{x - (-\frac{y}{t})}} \leq 
	\frac{1}{d}\norm{z}$$ и доказано $\norm{f} \leq \frac{1}{d}$. \\
	
	Чтобы доказать что $\norm{f} = \frac{1}{d}$, воспользуемся тем что $$\dist(x,L) = \inf_{y \in L}{\norm{x - y}} = d$$, а значит  $\exists\seq{y_n}{n\in\Natural}: \norm{x-y_n} \xrightarrow[n \to \infty]{} d$ и замечаем
	$$1 = f(x) - f(y_n) = f(x - y_n) \leq \norm{f}\norm{x - y_n} \xrightarrow[n \to \infty]{} \norm{f}d$$
	т.е. $\norm{f} \geq \frac{1}{d}$ чем полностью доказывается требуемое 3., остается лишь заметить что по теореме Хана-Банаха оператор продляется с $L_0$ до $X$ с сохранением всех трех свойств.
\end{proof}

\section{Общий вид функционалов в различных пространствах}

% Теорема Рисса об общем виде
\begin{theorem}[\textbf{Рисса об общем виде}]\label{th:2} 
	Пусть $H$ - г.п.
	\begin{enumerate}
		\item $(u \in H)\wedge(f(v)=(v,u), \forall v\in H) \Rightarrow (f\in H^*) \wedge (\norm{f}_{H^*}=\norm{u}_H)$
		\item $(f\in H^*)\Rightarrow (\exists! u\in H: (f(v)=(v,u),\forall v\in H)\wedge(\norm{f}_{H^*} = \norm{u}_H))$
	\end{enumerate}
\end{theorem}
\begin{proof}
	$ $ \\
	
	\textbf{1.} $f(v)=(v,u) \Rightarrow f\in L(H,\Real)$ т.к. по аксиомам скалярного произведения 
	$$(\alpha_1 v_1 + \alpha_2 v2, u) = (\alpha_1 v_1, u) + (\alpha_2 v_2, u) = \alpha_1 (v_1, u) + \alpha_2 (v_2, u)$$
	Из неравенства Коши-Буняковского получаем:
	$$|f(v)| = |(v,u)| \leq \norm{v}\norm{u},\forall v\in H \Rightarrow (f\in H^*) \wedge (\norm{f}_{H^*} \leq \norm{u}_H)$$	
	Теперь пусть $u \neq 0$ (случай $u = 0$ тривиален, это нулевой оператор), тогда положим 
	$$\hat{u} = \frac{u}{\norm{u}} \Rightarrow \norm{\hat{u}} = 1$$ 
	тогда $$f(\hat{u}) = (\frac{u}{\norm{u}}, u) = \frac{(u,u)}{\norm{u}} = \norm{u}$$
	и остается заметить что
	$$\norm{u}_H = |f(\hat{u})| \leq \norm{f}_{H^*} \norm{\hat{u}}_H = \norm{f}_{H^*} \Rightarrow \norm{f}_{H^*} = \norm{u}_{H}$$
	\newline
	
	\textbf{2.1.} Пусть $f \in H^*$, если $f = 0$, то $f(v)=(v,0),\forall v \in H$. Если $f \neq 0$, то 
	$$f \neq 0 \Rightarrow \ker{f} \text{ - подпр-во } \neq H \Rightarrow H = \ker{f} \oplus (\ker{f})^\perp$$ 
	Т.к. в этом случае $(\ker{f})^\perp \neq \{0\}$, то можем взять $x_1, x_2 \neq 0, x_1, x_2 \in (\ker{f})^\perp$ и рассмотреть следующий интересный элемент $H$:
	$$x = f(x_2)x_1 - f(x_1)x_2$$
	Понятно что $x \in (\ker{f})^\perp$, в тоже время
	$$f(f(x_2)x_1 - f(x_1)x_2) = f(x_2)f(x_1) - f(x_1)f(x_2) = 0$$ т.е. $x \in \ker{f}$. Как известно $M\cap M^\perp = \{0\}$, так что
	$$f(x_1)x_2 - f(x_2)x_1 = 0, f(x_1) \neq 0, f(x_2) \neq 0 \text{ т.к. не принадлежат ядру }$$
	Мы получили что любые два элемента $(\ker{f})^\perp$ линейно зависимы, а это означает что $\dim{(\ker{f})^\perp} = 1$ - ортогональное дополнение ядра является прямой. \\
	
	\textbf{2.2.} Таким образом $$\exists u_0 \neq 0: (\ker{f})^\perp = \setbuild{u\in H}{\exists t \in \Real: u = t u_0}$$
	и с учетом того что $H = \ker{f} \oplus (\ker{f})^\perp$
	$$\forall v \in H \: \exists!w\in\ker{f} \: \exists!t\in\Real: v = w + tu_0$$ 
	Замечаем
	$$f(v) = f(w) + tf(u_0) = 0 + tf(u_0) \text{ т.к. } w\in\ker{f} $$
	причем т.к. $u_0 \in (\ker{f})^\perp$, то $f(u_0) \neq 0$ и мы получили явное выражение для $t$:
	$$t = \frac{f(v)}{f(u_0)} $$
	
	\textbf{2.3.} Итого $v = w + \frac{f(v)}{f(u_0)} u_0$, посчитаем скалярное произведение:
	$$(v, u_0) = (w, u_0) + \frac{f(v)}{f(u_0)} (u_0, u_0) = f(v) \frac{\norm{u_0}^2}{f(u_0)}$$
	здесь следует напомнить что $(w, u_0) = 0$ т.к. $u_0 \in (\ker{f})^\perp$, а $w \in \ker{f}$
	Теперь остается принять 
	$$u = \frac{u_0}{\norm{u_0}^2} f(u_0) \text{ т.к. тогда } (v, u) = (v, u_0) \frac{f(u_0)}{\norm{u_0}^2} = f(v), \forall v \in H$$
	
	\textbf{2.4.} Когда такой элемент $u$ найден, можно воспользоваться \textbf{1.} и получить $\norm{f}_{H^*} = \norm{u}_H$ т.е. теорема почти доказана, но нам нужна единственность. Пусть 
	$$u_1, u_2 \in H: f(v) = (v, u_1) = (v, u_2), \forall v\in H$$
	, но тогда
	$$(v, u_1 - u_2) = 0,\forall v \in H \Rightarrow (u_1 - u_2, u_1 - u_2) = 0 \Rightarrow u_1 = u_2$$
	
	
\end{proof}
\begin{corollaryth}
	На первый взгляд кажется что эта теорема о том что на гильбертовом пространстве $H$ других линейных ограниченных функционалов кроме как скалярных произведений на некоторых элемент $u\in H$ - нет.
\end{corollaryth}
\begin{corollaryth}
	Тем не менее, её смысл куда глубже, он заключается в том что $H^*$ изометрически изоморфно $H$. Действительно: пусть $F: H^* \to H$ определяемое как $f \xrightarrow{F} u, \text{ где } u \in H: f(v) = (v,u)$ вывод теоремы \textbf{1} о том что это отображение сюръективно, а \textbf{2} о том что оно инъективно, а значит оно взаимно-однозначно. Изометричность видно из соотношений $\norm{f}_{H^*} = \norm{u}_H$ в обоих выводах теоремы, это значит что $\norm{f}_{H^*} = \norm{F(f)}_H$. Линейность $F$ очевидна и теперь можно считать что $H^* \sim H$
\end{corollaryth}

% Теорема Рисса в l^p
\begin{theorem}[\textbf{Рисса в $l^p$}]\label{th:3}
	Пусть $1 < p < \infty, q = \frac{p}{p-1}$ ($q$ - т.н. \textit{сопряженный показатель} т.е. такой что $\frac{1}{p} + \frac{1}{q} = 1$)
	
	\begin{enumerate}
		\item $(\xi \in l^p) \wedge \left(f(x) = \sum\limits_{k=1}^{\infty}{x_k \xi_k}, \forall x \in l^p\right) \Rightarrow
		(f \in \conj{l^p}) \wedge \left(\norm{f}_{\conj{l^p}} = \norm{\xi}_{l^q}\right)$
		\item $(f \in \conj{l^p}) \Rightarrow 
		\exists!\xi\in l^q: (f(x) = \sum\limits_{k=1}^{\infty}{x_k \xi_k}, \forall x \in l^p) \wedge \left(\norm{f}_{\conj{l^p}} = \norm{\xi}_{l^q}\right)$
	\end{enumerate}
\end{theorem}
\begin{proof}
	$ $ \\
	
	\textbf{1.} Первым делом покажем что $f$ определен для любого $x \in l^p$:
	$$f(x) = \sum\limits_{k=1}^{\infty}{x_k \xi_k} \leq \sum\limits_{k=1}^{\infty}{|x_k \xi_k|} \overset{\text{н-во Гельдера}}{\leq} 
	\left(\sum\limits_{k=1}^{\infty}{|x_k|^p}\right)^\frac{1}{p} \left(\sum\limits_{k=1}^{\infty}{|\xi_k|^q}\right)^\frac{1}{q} = \norm{x}_{l^p} \norm{\xi}_{l^q} $$
	Линейность очевидна по определению, оценкой выше доказывается ограниченность т.к. доказали что 
	$$|f(x)| \leq \norm{\xi}_{l^q} \norm{x}_{l^p}, \forall x \in l^p \Rightarrow
	(f \in \conj{l^p}) \wedge (\norm{f}_{\conj{l^p}} \leq \norm{\xi}_{l^q})$$
	, отдельно отметим неравенство
	\begin{equation}\label{eq:2}
		\norm{f}_{\conj{l^p}} \leq \norm{\xi}_{l^q}
	\end{equation}
	и докажем что имеет место строгое равенство, для этого нам потребуется специальный элемент:
	$$\tilde{x} = (\tilde{x}_1, ..., \tilde{x}_k, ...), \text{ где } 
	\tilde{x}_k = \frac{\sign{\xi_k} |\xi_k|^{q-1}}{\norm{\xi}_{l^q}^{\frac{q}{p}}}, \forall k \in \Natural$$
	Следует заметить что для данного определения предполагаем $\xi \neq 0$ т.к. случай $\xi = 0$ соответствует нулевому функционалу и тривиален. Убедимся что $\tilde{x} \in l^p$:
	$$|\tilde{x}_k|^p = \frac{|\xi_k|^{(q-1)p \ = \ q}}{\norm{\xi}_{l^q}^q}, \forall k \in \Natural \Rightarrow \sum\limits_{k=1}^{\infty}{|\tilde{x}_k|^p} =  \frac{\sum\limits_{k=1}^{\infty}{|\xi_k|^q}}{\norm{\xi}_{l^q}^q} = 1$$
	, т.е. действительно $\tilde{x} \in l^p$ и $\norm{\tilde{x}}_{l^p} = 1$, наконец с учетом того что $x\sign{x} = |x|$, получаем:
	\begin{align*}
		&f(\tilde{x}) \leq \norm{f}_{\conj{l^p}}\norm{\tilde{x}}_{l^p} = \norm{f}_{\conj{l^p}} \\ 
		&f(\tilde{x}) = \frac{\sum\limits_{k=1}^{\infty}{\sign{\xi_k}|\xi_k|^{q-1}\xi_k}}{\norm{\xi}_{l^q}^{\frac{q}{p}}} = \frac{\sum\limits_{k=1}^{\infty}{|\xi_k|^q}}{\norm{\xi}_{l^q}^{\frac{q}{p}}} = \frac{\norm{\xi}_{l^q}^q}{\norm{\xi}_{l^q}^{\frac{q}{p}}} = \norm{\xi}_{l^q}^{q(1 - \frac{1}{p}) \ = \ 1} = \norm{\xi}_{l^q}
	\end{align*}
	, т.е. $\norm{\xi}_{l^q} \leq \norm{f}_{\conj{l^p}}$ и с учетом \eqref{eq:2} имеем $\norm{\xi}_{l^q} = \norm{f}_{\conj{l^p}}$ т.е. первая часть полностью доказана.
	\\
	
	\textbf{2.} Пусть $f \in \conj{l^p}$ и пусть $e^k = (0, ..., 0, 1, 0, ...)$, где $1$ стоит на $k$-ой позиции, возьмем $\xi_k = f(e^k), \forall k \in \Natural$. Рассмотрим
	$$x = (x_1, ..., x_n, x_{n+1}, ...) \in l^p \text{ и пусть } x^{(n)} = (x_1, ..., x_n, 0, ...) = \sum_{k=1}^{n}{x_k e^k}$$
	, если теперь обозначить $\xi^{(n)} = (\xi_1, ..., \xi_n, 0, ...)$, то получим
	$$f(x^{(n)}) = \sum_{k=1}^{n}{x_k f(e^k)} = \sum_{k=1}^{n}{x_k \xi_k} = \sum_{k=1}^{\infty}{ x^{(n)}_k \xi^{(n)}_k }$$
	Аналогично первой части рассмотрим: $\tilde{x}^{(n)}, \forall n \in \Natural$
	\begin{equation}\label{eq:3}
		\tilde{x}_k^{(n)} = \frac{\sign{\xi_k^{(n)}} \left|\xi_k^{(n)}\right|^{q-1}}{\norm{\xi^{(n)}}_{l^q}^{\frac{q}{p}}}, \text{ и соотв. } \norm{\tilde{x}^{(n)}}_{l^p} = 1
	\end{equation}
	, и наконец (опять таки аналогично предыдущей части)
	\begin{align*}
		&\sum_{k=1}^{\infty}{ \tilde{x}^{(n)}_k \xi^{(n)}_k } = \norm{\xi^{(n)}}_{l^q} \\
		&\sum_{k=1}^{\infty}{ \tilde{x}^{(n)}_k \xi^{(n)}_k } = 
		\sum_{k=1}^{n}{ \tilde{x}^{(n)}_k \xi^{(n)}_k } = f(x^{(n)}) \leq 
		\norm{f}_{\conj{l^p}} \norm{\tilde{x}^{(n)}}_{l^p} \overset{~\eqref{eq:3}}{=} \norm{f}_{\conj{l^p}}
	\end{align*}
	, таким образом
	$$\norm{\xi^{(n)}}_{l^q}^q = \sum_{k=1}^{\infty}{|\xi_k^{(n)}|^q} \overset{\text{по опред. } \xi^{(n)}}{=} \sum_{k=1}^{n}{|\xi_k|^q} \leq \norm{f}_{\conj{l^p}}^q$$
	т.е. частичные суммы ряда оцениваются одним и тем же числом, соответственно ряд сходится и $\xi \in l^q$, причем 
	$$\norm{\xi}_{l^q} \leq \norm{f}_{\conj{l^p}}$$
	Заметим что т.к. $f \in \conj{l^p}$, то он непрерывен (поскольку ограничен)
	\begin{align*}
	f(x^{(n)}) &\to f(x) \\ &\wedge \\
	\sum_{k=1}^{n}{x_k \xi_k} &\to \sum_{k=1}^{\infty}{x_k \xi_k} \\ 
	\text{ (сходится абсолютно т.к. } &x \in l^p, \xi \in l^q \text{ + н-во Гельдера}) \\
	&\Rightarrow \\
	f(x) &= \sum_{k=1}^{\infty}{x_k \xi_k}, \forall x \in l^p
	\end{align*}
	Равенство норм доказывается аналогично первой части. Докажем единственность: пусть существует еще один
	$$\eta \in l^q: f(x) = \sum_{k=1}^{\infty}{x_k \eta_k}, \forall x \in l^p$$
	, тогда по построению $\xi$:
	$$\xi_j = f(e^j) = \sum_{k=1}^{\infty}{e^j_k \eta_k} = \eta_j, \forall j \in \Natural$$
	что и означает $\xi = \eta$, теорема доказана.
	
\end{proof}
\begin{corollaryth}
	Теорема устанавливает следующий изометрический изоморфизм: $\conj{l^p} \sim l^q$. Это оправдывает то, что мы назвали $q$ - сопряженным показателем. Если в \hyperref[th:2]{предыдущей теореме}, в случае гильбертовых пространств - $H^* \sim H$ т.е. сопряженное изоморфно самому ему, то в этой теореме Рисса уже другому пространству. Здесь следует заметить что среди $l^p$ гильбертовым является лишь $l^2$ и предыдущая теорема Рисса пересекается с этой т.к. если $p = q = 2$, то $\conj{l^2} \sim l^2$ уже по текущей теореме.
\end{corollaryth}


% Теорема Рисса в L^p
\begin{theorem}[\textbf{Рисса в $L^p$}]
	Пусть $1 < p < \infty, q = \frac{p}{p-1}$
	\begin{enumerate}
		\item $(g \in L^q (E)) \wedge \left(l(f) = \int\limits_{E}{fg dx}, \forall f \in L^p (E)\right) \Rightarrow \\ \Rightarrow \left(l \in \conj{L^p (E)}\right) \wedge \left(\norm{l}_{ \conj{L^p (E)} } = \norm{g}_{L^q (E)} \right)$
		\item $\left(l \in \conj{L^p (E)}\right) \Rightarrow \\ \Rightarrow \exists!g \in L^q (E): \left( l(f) = \int\limits_{E}{fg dx}, \forall f \in L^p (E)\right) \wedge \left(\norm{g}_{L^q (E)} = \norm{l}_{\conj{L^p (E)}} \right)$
	\end{enumerate}
\end{theorem}
\begin{proof}
	Без доказательства т.к. требуются свойства интеграла Лебега нам неизвестные.
\end{proof}
\begin{corollaryth}
	Теорема устанавливает следующий изометрический изоморфизм: $\conj{L^p (E)} \sim L^q$. На самом деле для всех используемых пространств имеются теоремы в духе теоремы Рисса (об общем виде) и каждая рассматривается отдельно.
\end{corollaryth}

\section{Слабая сходимость}
% Определение слабой сходимости
\begin{definition}
	Пусть $X$ - н.п., будем говорить что $x_n$ \textcolor{red}{сходится} к $x$ \textcolor{red}{слабо} и обозначать это как $x_n \weakto x$ (иногда пишут $x_n \overset{w}{\to} x$ от английского \textit{weak}) если $\forall f \in X^*: f(x_n) \to f(x)$
\end{definition}
\begin{corollarydf}
	В противопоставление, привычная нам сходимость в $X$: $x_n \to x \leftrightarrow \norm{x_n - x} \to 0$ называется \textcolor{red}{сильной}.
\end{corollarydf}

% Единственность предела при слабой сходимости
\begin{theorem}[\textbf{единственность предела } $\weakto$]
	Пусть $x_n \weakto x$ и $x_n \weakto y$ в $X$, тогда $x = y$
\end{theorem}
\begin{proof}
	По определению слабой сходимости:
	\begin{align*}
	&(f(x_n) \to f(x)) \wedge (f(x_n) \to f(y)), \forall f \in X^* \\ &\Rightarrow \\ 
	&f(x) = f(y), \forall f \in X^* \\  &\Leftrightarrow \\
	&f(x - y) = 0, \forall f \in X^* \\ &\overset{\text{\hyperref[corll:2]{следствие 1.2}}}{\Rightarrow} \\ 
	& x - y = 0
	\end{align*}
	, отсюда $x = y$, что и требовалось доказать.
\end{proof}

% Сильная влечет слабую
\begin{theorem}[\textbf{сильная $\Rightarrow$ слабая}]\label{th:4}
	Пусть $x_n \to x$ в $X$, тогда $x_n \weakto x$ в $X$
\end{theorem}
\begin{proof}
	Пусть $f \in X^*$, тогда $f$ - непрерывный, поэтому $(x_n \to x) \Rightarrow (f(x_n) \to f(x))$ т.е. $f(x_n) \to f(x), \forall f \in X^*$, ч.т.д.
\end{proof}
\begin{corollaryth}
	Рассмотрим пример в $l^2$, последовательность $e^i = (0, ..., 0, 1, 0, ...)$, где $1$ на $i$-ой позиции. Сильно она никуда не сходится. Пусть $f \in \conj{l^2}$, воспользуемся тут \hyperref[th:3]{теоремой Рисса}:
	$$\exists u \in l^2: f(e^i) = \sum_{k=1}^{\infty}{u_k e^i_k} = u_k \to 0$$
	, последнее верно т.к. $u \in l^2 \Rightarrow \sum\limits_{k=1}^{\infty}{|u_k|^2} < \infty$ и общий член ряда должен сходиться к нулю. Мы показали что $e^i \weakto 0$, значит слабая сходимость не всегда влечет сильную. Далее рассмотрим когда это верно.
\end{corollaryth}

% В конечномерных слабая и сильная совпадают
\begin{theorem}[\textbf{о $\weakto$ в $\Real^n$}]
	Пусть $X = \Real^n$, тогда $(x_k \weakto x \text{ в } \Real^n) \Rightarrow (x_k \to x \text{ в } \Real^n)$
\end{theorem}
\begin{proof}
	Пусть $e_1, ..., e_n$ базис в $\Real^n$, $x = \sum\limits_{i=1}^{n}{\alpha_i e_i} \in \Real^n$ и рассмотрим функционал $f(x) = \alpha_i$ для некоторого фиксированного $i \in \{1,...,n\}$, понятно что он линейный. В $\Real^n$ все нормы эквивалентны, поэтому можно взять \\ $\norm{x}_0 =\sum\limits_{i=1}^{n}{|\alpha_i|}$ и $|f(x)| = |\alpha_i| \leq \norm{x}_0, \forall x \in \Real^n$ т.е. этот функционал ограничен: $f \in \conj{\Real^n}$.
	\\ 
	
	Т.к. $x_k \weakto x$, то $f(x_k) \to f(x)$ т.е. для $x_k = \sum\limits_{i=1}^{n}{\alpha_{ki} e_i}$ имеем покоординатную сходимость $\alpha_{ki} \to \alpha_i, \forall i \in \{1, ..., n\}$, а в $\Real^n$ это и означает сильную сходимость $x_k \to x$, ч.т.д.
\end{proof}

% Слабая сходимость образов
\begin{theorem}[о слабой сходимости образов]
	Пусть $X, Y$ - н.п., $A \in \bounded{X}{Y}$ и $x_n \weakto x$ в $X$, тогда $Ax_n \to Ax$ в $Y$
\end{theorem}
\begin{proof}
	Требуется доказать что для произвольного $f \in Y^*$, выполнено $f(Ax_n) \to f(Ax)$, это наша цель. Рассмотрим функционал $\varphi$ определенный как суперпозиция $\varphi(x) = f(Ax), \forall x \in X$. Суперпозиция линейных операторов всегда является линейным оператором, давайте это покажем:
	\begin{multline*}
		\varphi(\alpha_1 x_1 + \alpha_2 x_2) = f(A(\alpha_1 x_1 + \alpha_2 x_2)) = f(\alpha_1 Ax_1 + \alpha_2 Ax_2) = \\ = \alpha_1 f(Ax_1) + \alpha_2 f(Ax_2) = \alpha_1 \varphi(x_1) + \alpha_2 \varphi(x_2)
	\end{multline*}
	С ограниченностью еще проще:
	$$|\varphi(x)| = |f(Ax)| \leq \norm{f}_{Y^*} \norm{Ax}_Y \leq \norm{f} \norm{A} \norm{x}, \forall x \in X$$
	Итого $\varphi \in X^*$, но тогда по определению слабой сходимости $x_n \weakto x$ имеем:
	$$\varphi(x_n) \to \varphi(x) \Leftrightarrow f(Ax_n) \to f(Ax)$$
	и видим что это то что нам и нужно.
\end{proof}

% Слабая полунепрерывность нормы
\begin{theorem}[слабая полунепрерывность нормы]
	Пусть $X$ - н.п., $x_n \weakto x$, тогда $\norm{x} \leq \lowlim\limits_{n \to \infty}{\norm{x_n}}$
\end{theorem}
\begin{proof}
	Пусть $d = \lowlim\limits_{n \to \infty}{\norm{x_n}}$, тогда т.к. нижний предел обязательно реализуется на какой-то последовательности: $\exists{x_{n_k}} \subset{x_n}: \norm{x_{n_k}} \to d$. Вспоминаем \hyperref[corll:1]{следствие 1} теоремы Хана-Банаха (для элемента $x$), по которому \\ $\exists f \in X^*: (\norm{f} = 1) \wedge (f(x) = \norm{x})$. Т.к. $x_n \weakto x$, элементарно показывается что для подпоследовательности аналогично $x_{n_k} \weakto x$, значит 
	$$f(x_{n_k}) \to f(x) = \norm{x}$$
	по определению слабой сходимости.
	
	Остается провести оценку:
	$$f(x_{n_k}) \leq |f(x_{n_k})| \leq \norm{f} \norm{x_{n_k}} \to d$$
	и с учетом того что $f(x_{n_k}) \to \norm{x}$ получаем
	$$\norm{x} \leq \lowlim\limits_{n \to \infty}{\norm{x_n}} = d$$
	, что и требовалось.
\end{proof}

% Введение в теорию двойственности
\section{Введение в теорию двойственности}
Пусть $X$ - б.п., возьмем какой-нибудь элемент $x \in X$ и рассмотрим функционал:
\begin{equation}\label{eq:5}
	g_{x}: X^* \to \Real \ \ g_{x}(f) = f(x), \forall f \in X^*
\end{equation}
изучим его свойства, для начала линейность:
\begin{multline*}
		g_{x}(\alpha_1 f_1 + \alpha_2 f_2) = (\alpha_1 f_1 + \alpha_2 f_2)(x) = \alpha_1 f_1 (x) + \alpha_2 f_2 (x) = \alpha_1 g_{x}(f_1) + \alpha_2 g_{x}(f_2)
\end{multline*}
таким образом действительно $g \in \linear{X^*}{\Real}$, еще проще доказывается ограниченность:
\begin{equation*}
	|g_{x}(f)| = |f(x)| \leq \norm{f}_{X^*} \norm{x}_{X}, \forall f \in X^*
\end{equation*}
итого $g_{x} \in \conj{X^*} = X^{**}$ и получили оценку для нормы 
\begin{equation}\label{eq:4}
	\norm{g_{x}}_{X^{**}} \leq \norm{x}_X
\end{equation}

Покажем что здесь на самом деле строгое равенство. Если $x = 0$, то \\ $g_{x}(f) = f(0) = 0, \forall f \in X^*$ и это нулевой функционал для которого равенство \eqref{eq:4} очевидно. Если $x \neq 0$, то то \hyperref[corll:1]{следствию 1} теоремы Хана-Банаха $\exists f_0 \in X^*: (f_0 (x) = \norm{x}_X) \wedge (\norm{f_0}_{X^*} = 1)$, тогда 
\begin{align*}
	g_{x} (f_0) &\leq \norm{g}_{X^{**}} \norm{f_0}_{X^*} = \norm{g}_{X^{**}} \\
	g_{x} (f_0) &= f_0 (x) = \norm{x}_X
\end{align*}
, отсюда $\norm{x}_X \leq \norm{g}_{X^{**}}$ и с учетом \eqref{eq:4}:  $\norm{g}_{X^{**}} = \norm{x}_X$.

Итого, мы установили что каждый элемент $X$ задает на сопряженном пространстве некоторый линейный ограниченный функционал и норма этого функционала равна норме элемента.

Покажем теперь что разные элементы задают разные функционалы. \begin{multline*}
	\left(g_{x_1} = g_{x_2}\right) \Leftrightarrow \left(g_{x_1}(f) = g_{x_2}(f), \forall f \in X^*\right)  \Leftrightarrow \left(f(x_1) = f(x_2), \forall f \in X^*\right) \Leftrightarrow \\ \Leftrightarrow \left(f(x_1 - x_2) = 0, \forall f \in X^* \right) \overset{\hyperref[corll:2]{\text{сл. 2 Т. Х-Б}}}{\Rightarrow} \left(x_1 = x_2\right)
\end{multline*}
следовательно между $X$ и некоторым подмножеством $R \subset X^{**}$ второго сопряженного пространства имеется взаимно-однозначное соответствие: $X \overset{g}{\leftrightarrow} R$. 

Покажем что это соответствие является изоморфизмом: пусть $x_1 \leftrightarrow g_{x_1}, \\ x_2 \leftrightarrow g_{x_2}$, тогда $\alpha_1 x_1 + \alpha_2 x_2 \leftrightarrow g_{\alpha_1 x_1 + \alpha_2 x_2}$ и рассмотрим этот функционал подробнее:
\begin{multline*}
	g_{\alpha_1 x_1 + \alpha_2 x_2}(f) = f(\alpha_1 x_1 + \alpha_2 x_2) = \alpha_1 f(x_1) + \alpha_2 f(x_2) = \alpha_1 g_{x_1}(f) + \alpha_2 g_{x_2}(f) = \\ = (\alpha_1 g_{x_1} + \alpha_2 g_{x_2})(f), \forall f \in X^*
\end{multline*}
и доказано равенство $g_{\alpha_1 x_1 + \alpha_2 x_2} = \alpha_1 g_{x_1} + \alpha_2 g_{x_2}$. 

Итого $g$ - изометрический изоморфизм (то что норма сохраняется было доказано ранее). Т.к. $X$ - б.п., то изометрически-изоморфное ему $R$ тоже б.п., а $X^{**}$ всегда б.п. как и $X^*$. Теперь вспоминаем что подмножество банахова пространства само является банаховым пространством относительно индуцированной нормы если и только если это подпространство т.е. $R$ - подпространство $X^{**}$. Введем теперь важное
\begin{definition}
	Банахово пространство $X$ называется \textcolor{red}{рефлексивным} если $R = X^{**}$ т.е. 
	$$\forall g \in X^{**} \exists! x\in X: g(f) = f(x), \forall f \in X^*$$
\end{definition}
\begin{corollarydf}
	Мы установили очень интересный факт: если у нас есть некоторое б.п. (даже н.п.) $X$, то его сопряженное $X^*$ это функционалы действующие на $X$, но оказывается сами элементы $X$ также являются функционалами над $X^*$ действующие по формуле \eqref{eq:5}. В зависимости от того рефлексивное пространство или нет, все функционалы на $X^*$ определяются элементами из $X$ или нет. Это и называется \textcolor{red}{двойственностью}, иногда чтобы её подчеркнуть пишут $f(x) = \left<f, x\right>$, чтобы показать равноправность.
\end{corollarydf}

% Важное следствие
\begin{theorem}[об ограниченности слабосходящихся $\{x_n\}_{n\in\Natural}$]
	Пусть $X$ - б.п. и $x_n \weakto x$ в $X$, тогда $\exists M \geq 0: \norm{x_n} \leq M, \forall n \in \Natural$
\end{theorem}
\begin{proof}
	$x_n \weakto x$ означает что $f(x_n) \to f(x), \forall x \in X^*$, а с учетом только что введенных обозначений в рамках теории двойственности это тоже самое что $g_{x_n} (f) \to g_{x} (f), \forall f \in X^*$. Теперь вспоминаем теорему Банаха-Штейнгауза, она говорит что если есть сильная операторная сходимость, то $$\sup\limits_{n \in \Natural}{\norm{g_{x_n}}_{X^{**}}} < \infty$$ что равносильно 
	$$\exists M \geq 0: \norm{g_{x_n}}_{X^{**}} = \norm{x_n}_X \leq M, \forall n \in \Natural $$
\end{proof}

% Определение сужения
\begin{definition}
	Пусть $L$ - подпространство в б.п. $X$ и $F \in X^*$, \textcolor{red}{сужением} $F$ на $L$ будем называть:
	\begin{equation*}
		F|_L : L \to \Real \ \ F|_L (y) = F(y), \forall y \in L
	\end{equation*}
\end{definition}
\begin{corollarydf}\label{corll:3}
	Пусть $y_1, y_2 \in L$, тогда $\alpha_1 y_1 + \alpha_2 y_2 \in L$ т.к. $L$ - подпространство и:
	\begin{multline*}
		F|_L (\alpha_1 y_1 + \alpha_2 y_2) = F(\alpha_1 y_1 + \alpha_2 y_2) = \alpha_1 F(y_1) + \alpha_2 F(y_2) = \alpha_1 F|_L (y_1) + \alpha_2 F|_L (y_2)
	\end{multline*}
	где мы пользовались определением сужения и линейностью $F \in X^*$, таким образом $F|_L \in \linear{L}{\Real}$. Теперь докажем ограниченность:
	$$|F|_L (y)| = |F(y)| \leq \norm{F}_{X^*} \norm{y}_{X} = \norm{F}_{X^*} \norm{y}_L, \forall y \in L$$
	здесь мы пользовались тем что на подпространстве $L$ норма индуцирована и ограниченностью $F$. Итого $F|_L \in L^*$ и $\norm{F|_L}_{L^*} \leq \norm{F}_{X^*}$. 
\end{corollarydf}

\begin{theorem}[о рефлексивности подпространства]
	Пусть $X$ - рефлексивное б.п. и $L$ - подпространство в $X$ (заметим что оно тоже б.п.), тогда $L$ - рефлексивное пространство.
\end{theorem}
\begin{proof}
	Для удобства договоримся что функционалы из $X^*, X^{**}$ - большими буквами ($G, F$), а из $L^*, L^{**}$ - маленькими ($g, f$).
	
	\textbf{1.} Пусть $g \in L^{**}$, мы хотим доказать что $\exists x \in L: g(f) = f(x), \forall f \in L^*$. Введем функционал 
	$$G: X^* \to \Real \ \ G(F) = g(F|_L), \forall F \in X^*$$
	и изучим его свойства. Он линейный т.к.:
	\begin{multline*}
		G(\alpha_1 F_1 +\alpha_2 F_2) = g((\alpha_1 F_1 + \alpha_2 F_2)|_L) = g(\alpha_1 F_1|_L + \alpha_2 F_2 |_L) = \\ = \alpha_1 g(F_1|_L) + \alpha_2 g(F_2|_L) = \alpha_1 G(F_1) + \alpha_2 G(F_2)
	\end{multline*}
	здесь мы пользовались тем очевидным свойством что сужение линейной комбинации функционалов это линейная комбинация сужений и доказали $G \in \linear{X^*}{\Real}$. Точно также покажем ограниченность:
	$$|G(F)| = |g(F|_L)| \leq \norm{g}_{L^{**}} \norm{F|_L}_{L^*} \overset{\hyperref[corll:3]{\text{опред.}}}{\leq} \norm{g}_{L^{**}} \norm{F}_{X^*}, \forall F \in X^*$$
	значит $G \in X^{**}$ и кроме того получили оценку для нормы $\norm{G}_{X^{**}} \leq \norm{g}_{L^{**}}$
	
	\textbf{2.} Т.к. $X$ рефлексивно, то $\exists! x \in X: G(F) = F(x), \forall F \in X^*$. Докажем что при этом $x \in L$. Пусть $x \notin L$, тогда $x$ не может быть предельной точкой $L$ (т.к. подпространства всегда замкнутые), т.е. он должен быть отделен: $\dist(x, L) > 0$ и можно применить \hyperref[corll:4]{следствие 3} теоремы Хана-Банаха: $\exists F_0 \in X^*: (F_0 (x) = 1) \wedge (F(y) = 0, \forall y \in L)$. Очевидно $F_0 |_L (y) = 0, \forall y \in L$. Заметим:
	\begin{align*}
		G(F_0) &= F_0 (x) = 1 \\
		G(F_0) &= g(F_0 |_L) = g(0) = 0
	\end{align*}
	получили противоречие и $x \in L$.
	
	\textbf{3.} Рассматриваем $g(f)$, где $f \in L^*$. По \hyperref[th:1]{теореме Хана-Банаха} $\exists F \in X^*$ такой что $F$ - продолжение $f$ т.е. $F|_L = f$, причем $\norm{F}_{X^*} = \norm{f}_{L^*}$, но тогда:
	$$g(f) = g(F|_L) = G(F) = F(x) \overset{x \in L}{=} f(x), \forall f \in L^*$$
	нашу цель мы выполнили и $L$ - рефлексивно.
\end{proof}

\section{Слабая сходиомость в гильбертовых пространствах}
Гильбертовы пространства частный случай банаховых, поэтому все что мы проходили до этого (для б.п.) здесь остается в силе. Т.к. они имеют много специфических свойств, можно доказать куда больше чем в общем случае. Вспомним этот самый общий случай:
\begin{equation*}
	(H \text{ - г.п. }, u_n \weakto u \text{ в } H) \Leftrightarrow
	\left(f(u_n) \to f(u), \forall f \in H^*\right) \overset{\hyperref[th:2]{т. Рисса}}{\Leftrightarrow} \left((u_n, w) \to (u, w), \forall w \in H\right)
\end{equation*}

\begin{proposition}
	$(u_n \to u) \Leftrightarrow (u_n \weakto u) \wedge (\norm{u_n} \to \norm{u})$
\end{proposition}
\begin{proof}
	Необходимость очевидна т.к. для сходимостей \hyperref[th:4]{сильная $\Rightarrow$ слабая} и норма непрерывна. Докажем достаточность:
	\begin{equation*}
		\norm{u_n - u}^2 = \norm{u_n}^2 - 2(u_n, u) + \norm{u}^2 \to \norm{u}^2 - 2(u, u) + \norm{u}^2 = 0
	\end{equation*}
	где мы пользовались тем что $u_n \weakto u \Rightarrow (u_n, u) \to (u, u)$ ($w = u$ в определении)
\end{proof}

% Принцип выбора
Теперь докажем важнейший результат этой теории. Вспомним принцип Больцано-Вейерштрасса в $\Real^n$ о том что из любой ограниченной последовательности можно выделить сходящуюся подпоследовательность. Хотелось бы иметь некоторый аналог в бесконечномерных пространствах, но для сильной сходимости это неверно, а вот для слабой имеем:
\begin{theorem}[принцип выбора]
	Пусть $H$ - г.п. и \\ $\seq{u_n}{n \in \Natural} \subset H: \norm{u_n} \leq M, \forall n \in \Natural$, тогда 
	$$\exists \seq{u_{n_k}}{k \in \Natural} \subset \seq{u_n}{n \in \Natural} \ \exists u \in H: u_{n_k} \weakto u \text{ в } H$$
\end{theorem}
\begin{proof}
	$ $ \newline
	
	\textbf{1.} Для $\seq{u_n}{n \in \Natural}$ получим числовую последовательность $\seq{(u_n, u_1)}{n\in\Natural} \subset \Real$. Она ограничена т.к. по неравенству Коши-Буняковского:
	$$|(u_n, u_1)| \leq \norm{u_n} \norm{u_1} \leq M^2$$
	Значит $\exists \seq{u_n^{(1)}}{n\in\Natural} \subset \seq{u_n}{n\in\Natural}: \seq{(u_n^{(1)}, u_1)}{n\in\Natural} \text{ - сходится}$.
	
	Для $\seq{u_n^{(1)}}{n\in\Natural}$ получим $\seq{(u_n^{(1)}, u_2)}{n\in\Natural} \subset \Real$. Аналогично она будет ограничена (той же константой $M^2$) и можно выделить подпоследовательность 
	$$\seq{u_n^{2}}{n\in\Natural} \subset \seq{u_n^{(1)}}{n\in\Natural} \subset \seq{u_n}{n \in \Natural}: (u_n^{(2)}, u_2) \text{ - сходится}$$
	причем $\seq{(u_n^{(2)}, u_1)}{n\in\Natural}$ (уже скалярное произведение на $u_1$, а не на $u_2$) - подпоследовательность сходящейся последовательности $\seq{(u_n^{(1)}, u_1)}{n\in\Natural}$ которая тоже сходится.
	
	Продолжая процесс получим 
	\begin{equation*}
		\seq{u_n^{(m)}}{n \in \Natural} \subset \seq{u_n^{(m-1)}}{n \in \Natural} \subset \cdots \subset \seq{u_n^{(1)}}{n \in \Natural} \subset \seq{u_n}{n \in \Natural}
	\end{equation*}
	причем по выкладкам аналогичным ранее для $\seq{u_n^{(2)}}{n\in\Natural}$:
	\begin{equation*}
		\seq{(u_n^{(m)}, u_k)}{n\in\Natural} \text{ - сходится}, \forall k \leq m
	\end{equation*}
	
	\textbf{2.} Берем вот такую <<диагональную>> последовательность $\seq{u_n^{(n)}}{n\in\Natural} \subset \seq{u_n}{n\in\Natural}$. Кроме того, начиная с номера $m$ все элементы этой последовательности выбирались из ${u_n^{(m)}}$ (из неё самой и из подпоследовательностей):
	\begin{equation*}
		\seqbound{u_n^{(n)}}{n=m}{\infty} \subset \seq{u_n^{(m)}}{n\in\Natural}
	\end{equation*}
	, а это значит что $\seq{(u_n^{(n)}, u_m)}{n\in\Natural}, \forall m \in \Natural$ - сходится т.к. сходимость не зависит от конечного числа членов. Итого, если ранее построенные подпоследовательности давали сходящиеся числовые будучи домноженными на элементы с индексом не более чем $m$, то вот эта новая <<диагональная>> последовательность обеспечивает сходимость при домножении на любой элемент исходной.
	
	\textbf{3.} Рассмотрим $L = \Lin(\seq{u_n}{n\in\Natural})$. Понятно что это линейное многообразие в $H$ и если $v \in L$, то $\seq{(u_n^{(n)}, v)}{n\in\Natural} \text{ - сходится}$ т.к. $v$ - конечная линейная комбинация элементов последовательности для которых эта сходимость обеспечена в пункте \textbf{2}. 
	
	\textbf{4.} Пусть теперь $v \in \bar{L}$. Покажем что $\seq{(u_n^{(n)}, v)}{n\in\Natural} \text{ - cходится}$, для этого докажем фундаментальность, пусть $w \in L$ и:
	\begin{multline}\label{eq:6}
		|(u_n^{(n)}, v) - (u_m^{(m)}, v)| = |(u_n^{(n)} - u_m^{(m)}, v)| \leq |(u_n^{(n)} - u_m^{(m)}, v - w)| + |(u_n^{(n)} - u_m^{(m)}, w)| \leq \\ \leq \norm{u_n^{(n)} - u_m^{(m)}}\norm{v - w} + |(u_n^{(n)} - u_m^{(m)}, w)|
	\end{multline}
	Заметим что 
	$$\norm{u_n^{(n)} - u_m^{(m)}} \leq \norm{u_n^{(n)}} + \norm{u_m^{(m)}} \leq 2M$$
	и т.к. $L$ плотно в $\bar{L}$, то $\forall \varepsilon > 0 \ \exists w \in L: \norm{v - w} < \frac{\varepsilon}{4M}$. Зафиксируем такое $w$ и т.к. $w\in L$, то по пункту \textbf{3} последовательность $(u_n^{(n)}, w)$ - сходится т.е. фундаментальна и $\exists N \in \Natural: |(u_n^{(n)} - u_m^{(m)}, w)| < \frac{\varepsilon}{2}, \forall n, m \geq N$, наконец из \eqref{eq:6} получаем что
	\begin{equation*}
		|(u_n^{(n)}, v) - (u_m^{(m)}, v)| < \varepsilon, \forall n,m \geq N
	\end{equation*}
	т.е. последовательность $\seq{(u_n^{(n)}, v)}{n\in\Natural}$ фундаментальна и сходится т.к. $\Real$ - полное м.п.
	
	\textbf{5.} Пусть $v \in H$, покажем что $\seq{(u_n^{(n)}, v)}{n\in\Natural} \text{ - cходится}$ (уже более общий случай). Т.к. $\bar{L}$ - замкнутое линейное многообразие т.е. подпространство, то $H = \bar{L} \oplus \bar{L}^\perp$, а значит 
	$$\forall v \in H \ \exists! l \in \bar{L}, l^\perp \in \bar{L}^\perp: v = l + l^\perp$$
	, но тогда
	$$(u_n^{(n)}, v) = (u_n^{(n)}, l) + (u_n^{(n)}, l^\perp) \overset{u_n^{(n)} \in \bar{L}}{=} (u_n^{(n)}, l) + 0 = (u_n^{(n)}, l) \text{ - cходится}$$
	т.к. $l \in \bar{L}$ и по пункту \textbf{4}.
	
	\textbf{6.} Наконец, построим функционал:
	\begin{equation*}
		f: H \to \Real \ \ f(v) = \lim_{n \to \infty}{(u_n^{(n)}, v)}, \forall v \in H
	\end{equation*}
	и посмотрим его свойства. Он линейный т.к.:
	\begin{multline*}
		f(\alpha_1 v_1 + \alpha_2 v_2) = \lim_{n\to\infty}{(u_n^{(n)}, \alpha_1 v_1 + \alpha_2 v_2)} = \lim_{n\to\infty}{\left(\alpha_1 (u_n^{(n)}, v_1) + \alpha_2 (u_n^{(n)},v_2)\right)} = \\ = 
		\alpha_1 \lim_{n\to\infty}{(u_n^{(n)}, v_1)} + \alpha_2 \lim_{n\to\infty}{(u_n^{(n)}, v_2)} = \alpha_1 f(v_1) + \alpha_2 f(v_2) 
	\end{multline*}
	еще проще показать ограниченность:
	\begin{equation*}
		|f(v)| = \left|\lim_{n\to\infty}{(u_n^{(n)}, v)}\right| = \lim_{n\to\infty}{|(u_n^{(n)}, v)|} \leq \lim_{n\to\infty}{\norm{u_n^{(n)}}\norm{v}} \leq M\norm{v}
	\end{equation*}
	Таким образом $f \in H^*$ и по \hyperref[th:2]{теореме Рисса} $$\exists! u \in H: f(v) = (u, v), \forall v \in H \Rightarrow \lim_{n\to\infty}{(u_n^{(n)}, v)} = (u, v), \forall v \in H \Rightarrow u_n^{(n)} \weakto u \text{ в } H$$
	, та самая подпоследовательность которая нам была нужна, а именно $\seq{u_n^{(n)}}{n\in\Natural}$ построена и теорема доказана.
\end{proof}
\begin{corollaryth}
	Тот же самый результат справедлив в случае если $H$ - сепарабельное рефлексивное пространство. Аналогично в них будут справедливы аналоги теорем которые будут дальше в пределах параграфа.
\end{corollaryth}


\newpage
\renewcommand{\listtheoremname}{Список теорем и утверждений}
\listoftheorems[ignoreall, show={theorem,corollary}]

\end{document}
