\documentclass[12pt,a4paper]{article}
\usepackage[utf8]{inputenc}
\usepackage[english,russian]{babel}
\usepackage{hyperref}
\usepackage{amsthm,amssymb,amsfonts,amsmath}
\usepackage{xcolor}
\usepackage[left=3cm,right=3cm,
top=3cm,bottom=3cm,bindingoffset=0cm]{geometry}
\usepackage{thmtools}

\theoremstyle{definition}
\newtheorem{theorem}{Теорема}
\newtheorem{definition}{Определение}[section]
\newtheorem{corollarydf}{Замечание}[definition]
\newtheorem{corollaryth}{Замечание}[theorem]
\newtheorem{corollary}{Следствие}[theorem]

\newcommand{\Real}{\mathbb{R}}
\newcommand{\Cmplx}{\mathbb{C}}
\newcommand{\Natural}{\mathbb{N}}
\newcommand{\norm}[1]{\left\lVert#1\right\rVert}
\newcommand{\setbuild}[2]{\{#1\:|\:#2\}}
\DeclareMathOperator{\Lin}{Lin}
\DeclareMathOperator{\dist}{dist}
\newcommand{\tick}[1]{#1^{\prime}}
\newcommand{\dtick}[1]{#1^{\prime\prime}}
\newcommand{\bounded}[2]{\textrm{B}(#1, #2)}
\newcommand{\linear}[2]{\textrm{L}(#1, #2)}
\newcommand{\seq}[2]{\{#1\}_{#2}}

\hypersetup{
	colorlinks   = true, 
	urlcolor     = blue, 
	linkcolor    = blue, 
	citecolor   = red
}

%opening
\title{Линейные функционалы}
\date{}

\begin{document}

\maketitle

\begin{abstract}
	Конспект лекций по данному разделу, читаемых Сергеем Валерьевичем Лупуляком в СПбПУ. Видеозаписи:
	\begin{enumerate}
		\item \url{https://www.youtube.com/watch?v=NyO3tRUWof0}
		\item \url{https://www.youtube.com/watch?v=FkBqjrC1pjI}
	\end{enumerate}
	
\end{abstract}

\newpage

\section{Теорема Хана-Банаха}

% Определение сопряженного пространства
\begin{definition}
	Пусть $X$ - н.п. над $\Real(\Cmplx)$, пространством \textcolor{red}{сопряженным к $X$} называем $X^*=\bounded{X}{\Real}(\bounded{X}{\Cmplx})$. Причем $f\in X^*\Rightarrow D(f)=X$
\end{definition}
\begin{corollarydf}
	Ключевая в этом параграфе теорема даст своего рода обоснование указанному в определении равенству $X^*=\bounded{X}{\Real}$, хотя более строго следовало бы написать $X^*=\setbuild{f\in \bounded{X}{\Real}}{D(f)=X}$ т.к. область определения ограниченного линейного функционала вообще говоря может быть произвольным линейным многообразием
\end{corollarydf}
\begin{corollarydf}
	Как известно если $Y$ - б.п., то $\bounded{X}{Y}$ - б.п. относительно нормы оператора $$\norm{f}_{X^*}=\sup_{\norm{x}\leq1}|f(x)|$$, а значит каким бы не было н.п. $X$, $X^*$ всегда б.п. (т.к. $\Real,\Cmplx$ - б.п.)
\end{corollarydf}

% Теорема Хана-Банаха
\begin{theorem}[\textbf{Хана-Банаха}]
	Пусть 
	\begin{enumerate}
		\item $X$ - н.п.
		\item $L\subset X$ - линейное многообразие
		\item $f\in \bounded{X}{\Real}, D(f) = L$
	\end{enumerate}
	тогда $\exists F\in X^*$ такой что
	\begin{enumerate}
		\item $F$ - продолжение $f$
		\item $\norm{F}_{X^*} = \norm{f}_{\bounded{X}{\Real}}$
	\end{enumerate}
\end{theorem}
\begin{proof}
	Докажем только для частного случая когда $X$ - сепарабельно. \newline \newline
	
	\textbf{1.} Пусть $x_0 \notin L$ (если такого не существует, то $L=X$ и утверждение тривиально) и пусть $P_{x_0} = \Lin{\{x_0\}} = 
	\setbuild{x\in X}{\exists\alpha\in\Real: x=\alpha x_0}$ (прямая натянутая на $x_0$). Пусть $L_0 = L + P_{x_0}$. \newline
	\newline
	
	\textbf{2.} Докажем что $L_0 = L\oplus P_{x_0}$. \newline
	Пусть $0=y_0 + tx_0$, где $y_0\in L$. Если $t\neq0$, то $x_0 = -\frac{y_0}{t} \in L$ - противоречие. Значит $t=0$ и соответственно $y_0 = 0$. Представление нуля единственно $0 = 0 + 0$, откуда теперь замечаем что если для некоторого $x\in L_0$: $x = y_1 + t_1 x_0 = y_2 + t_2 x_0$, то $0 = (y_1 - y_2) + (t_1 - t_2) x_0$ и очевидно $y_1 - y_2 \in L$, $(t_1 - t_2)x_0 \in P_{x_0}$ так что $y_1 = y_2 \wedge t_1 = t_2$. \newline
	\newpage
	
	\textbf{3.} Будем производить некоторые преобразования и оценки. Хоть и сложно сходу понять зачем, но увидим. Пусть $\tick{x}, \dtick{x} \in L$, тогда \newline\newline
	$f(\tick{x}) - f(\dtick{x}) = 
	f(\tick{x} - \dtick{x}) \leq |f(\tick{x} - \dtick{x})| \leq
	\norm{f}\norm{\tick{x} - \dtick{x}} =
	\norm{f}\norm{\tick{x} \pm x_0 - \dtick{x}} \leq \\
	\leq \norm{f}\norm{\tick{x} + x_0} + \norm{f}\norm{x_0 + \dtick{x}}$ \newline
	Итого 
	$$f(\tick{x}) - \norm{f}\norm{\tick{x} + x_0} \leq  f(\dtick{x}) + \norm{f}\norm{\dtick{x} + x_0}, \forall \tick{x},\dtick{x} \in L$$ 
	откуда
	$$\exists C \in \Real: \sup_{y \in L}{(f(y) - \norm{f}\norm{y + x_0})} \leq C \leq
	\inf_{y \in L}{(f(y) + \norm{f}\norm{y + x_0})}$$ \newline
	для последнего перехода важно было что в левой части только $\tick{x}$, а в правой только $\dtick{x}$, наконец замечаем
	$$f(y) - \norm{f}\norm{y + x_0} \leq C \leq f(y) + \norm{f}\norm{y + x_0}, \forall y \in L$$
	что равносильно
	\begin{equation}
		|f(y) - C| \leq \norm{f}\norm{y + x_0}, \forall y \in L \label{eq:1}
	\end{equation}
	\newline
	
	\textbf{4.} Пусть $x \in L_0$, тогда из \textbf{2} $\exists!{y \in L, t \in \Real}: x = y + tx_0$, соответственно можно задать $$F(x) = f(y) - tC$$, покажем что $F \in \linear{L_0, \Real}$. Действительно для $x_1, x_2 \in L_0$ имеем \newline $x_1 = y_1 + t_1 x_0, x_2 = y_2 + t_2 x_0$, где $y_1, y_2 \in L$ и $t_1, t_2 \in \Real$, тогда
	\begin{align*}
		\alpha_1 x_1 + \alpha_2 x_2 &= \alpha_1 (y_1 + t_1 x_0) + \alpha_2 (y_2 + t_2 x_0), \forall \alpha_1, \alpha_2 \in \Real  \\
		&\Leftrightarrow \\
		\alpha_1 x_1 + \alpha_2 x_2 &= (\alpha_1 y_1 + \alpha_2 y_2) + (\alpha_1 t_1 + \alpha_2 t_2)x_0, \forall \alpha_1, \alpha_2 \in \Real
	\end{align*}
	 Заметим что в $\alpha_1 y_1 + \alpha_2 y_2 \in L$ т.к. $L$ - линейное многообразие \newline и $\alpha_1 t_1 + \alpha_2 t_2 \in \Real$ так что с учетом того что $L_0 = L \oplus P_{x_0}$ и из определения $F$ получаем
	\begin{align*}
		F(\alpha_1 x_1 + \alpha_2 x_2) &= f(\alpha_1 y_1 + \alpha_2 y_2) - (\alpha_1 t_1 + \alpha_2 t_2)C, \forall \alpha_1, \alpha_2 \in \Real \\ &\Leftrightarrow \\    
		F(\alpha_1 x_1 + \alpha_2 x_2) &= \alpha_1 f(y_1) + \alpha_2 f(y_2) - \alpha_1 t_1 C - \alpha_2 t_2 C, \forall \alpha_1, \alpha_2 \in \Real \\ &\Leftrightarrow \\
		F(\alpha_1 x_1 + \alpha_2 x_2) &= \alpha_1 (f(y_1) - t_1 C) + \alpha_2 (f(y_2) - t_2 C), \forall \alpha_1, \alpha_2 \in \Real \\ &\Leftrightarrow \\
		F(\alpha_1 x_1 + \alpha_2 x_2) &= \alpha_1 F(x_1) + \alpha_2 F(x_2), \forall \alpha_1, \alpha_2 \in \Real
	\end{align*}
	Пусть $x\in L$, тогда $x = x + 0x_0 \in L_0$ и $F(x) = f(x) - 0C = f(x)$, так что $F$ - действительно продолжение $f$. Далее пусть $x \notin L$, тогда $t \neq 0$ и 
	$$|F(x)| = |f(y) - tC| = |t||f(\frac{y}{t}) - C| \leq^{\text{\eqref{eq:1}}} |t|\norm{f}\norm{\frac{y}{t} + x_0} = \norm{f}\norm{y + tx_0} = \norm{f}\norm{x}$$, так что вообще $\forall x \in L_0: \norm{F(x)} \leq \norm{f}\norm{x}$. Мы доказали что $F \in \bounded{L_0}{\Real}$, причем $\norm{F} \leq \norm{f} \Rightarrow \norm{F} = \norm{f}$ т.к. норма продолжения $f$ всегда не меньше нормы $f$.
	\newline
	
	\textbf{5.} По условию $X$ - сепарабельно т.е. $\exists \seq{x_k}{k \in \Natural} \subset X: X = \overline{\seq{x_k}{k \in \Natural}}$. \newline
	
	\noindent Пусть $x_1 \notin L, L_1 = L \oplus P_{x_1}$ и воспользовавшись \textbf{1-4} построим $f_1$ - продолжение $f$ на $L_1$ ($f_1 \in \bounded{L_1}{\Real}, \norm{f_1} = \norm{f}$). \newline
	
	\noindent Пусть $x_2 \notin L_1, L_2 = L_1 \oplus P_{x_2}$ и аналогично построим $f_2$ - продолжение $f_1$ на $L_2$. \newline
	
	\noindent Процесс продолжения приводит к последовательности вложенных линейных многообразий $L \subset L_1 \subset L_2 \subset ... \subset L_k \subset ...$. Если начиная с некоторого $K \in \Natural$ имеем $L_k = X, \forall k \geq K$, то теорема доказана. В противном случае пусть $$X_0 = \bigcup_{k=1}^{\infty}{L_k}$$, тогда заметим что $\seq{x_k}{k \in \Natural} \subset X_0$ и соответственно $\overline{X_0} = X$. \newline
	
	\noindent $x \in X_0 \Rightarrow \exists k \in \Natural: x \in L_k$, пусть $F(x) = f_k (x)$ и видим что $$|F(x)| = |f_k (x)| \leq \norm{f_k}\norm{x} = \norm{f}\norm{x}$$ т.е. $F \in \bounded{X_0}{\Real}$, а значит можно воспользоваться теоремой о продолжении ограниченного линейного отображения со всюду плотной областью определения и теорема доказана. 
\end{proof}
\begin{corollaryth}
	Возвращаясь к теории линейных операторов - ограниченный оператор можно было продлить со всюду-плотного линейного многообразия, а ограниченный функционал можно с произвольного. Однако, уже не единственным образом.
\end{corollaryth}

% Следствие 3
\begin{corollary}
	Пусть $x_0 \neq 0$, тогда \\ $\exists f\in X^*: (\norm{f}_{X^*} = 1) \wedge (f(x_0) = \norm{x_0})$
\end{corollary}
\begin{proof}
	Пусть $P_{x_0} = \setbuild{x\in X}{\exists t\in\Real: x = tx_0}$ и пусть $x\in P_{x_0}$, тогда $x=tx_0$ для некоторого $t$ и положим $f(x)=t\norm{x_0}$. Получили $f:P_{x_0} \to \Real$, теперь докажем что $f(x)\in \linear{P_{x_0}}{\Real}$ \\
	
	Действительно если $x_1,x_2 \in P_{x_0}$, то $\exists t_1,t_2\in\Real: (x_1 = t_1 x_0) \wedge (x_2 = t_2 x_0)$, а значит $$\alpha_1 x_1 + \alpha_2 x_2 = \alpha_1 t_1 x_0 + \alpha_2 t_2 x_0 = (\alpha_1 t_1 + \alpha_2 t_2)  x_0 \in P_{x_0}$$, и $$f(\alpha_1 x_1 + \alpha_2 x_2) = (\alpha_1 t_1 + \alpha_2 t_2)\norm{x_0} = \alpha_1 (t_1 \norm{x_0}) + \alpha_2 (t_2 \norm{x_0}) = \alpha_2 f(x_1) + \alpha_2 f(x_2)$$ \\
	
	Теперь заметим что $$|f(x)| = |t|\norm{x_0} = \norm{tx_0} = \norm{x}, \forall x \in P_{x_0}$$ таким образом мы доказали что $f\in\bounded{X}{\Real}$, $D(f) = P_{x_0}$ и $\norm{f} = 1$, а также как не трудно видеть $x_0 = 1\cdot x_0$ т.е. $f(x_0) = \norm{x_0}$ по определению. Все необходимые нам свойства выполнены, но этот функционал задан только на одномерном линейном многообразии. Вот теперь мы и применяем теорему Хана-Банаха, продляя его на все пространство и мы получили искомый $f\in X^*$
\end{proof}

% Следствие 2
\begin{corollary}
	Пусть $x\in X$ и $\forall f \in X^* : f(x)=0$, тогда $x = 0$
\end{corollary}
\begin{proof}
	Внимательно смотрим на предыдущее следствие. Если $x \neq 0$, то $\exists f \in X^*: f(x) = \norm{x} \neq 0$ что противоречит условию.
\end{proof}

% Следствие 3
\begin{corollary}
	Пусть $L\subset X$ - лин. многообразие, а $x\in X: \dist(x,L) = d > 0$, тогда $\exists f\in X^*$:
	\begin{enumerate}
		\item $f(x)=1$
		\item $f(y)=0,\forall y \in L$
		\item $\norm{f}_{X^*} = \frac{1}{d}$
	\end{enumerate}
\end{corollary}
\begin{proof}
	Пусть $L_0 = L \oplus P_{x}$, напомним что по определению прямой суммы $\forall z \in L_0 \exists!y\in L, t\in\Real: z = y + tx$. Определим функционал $f : L_0 \to \Real$ так: $f(z)=t$. Из определения $f$ сразу получаем что $$f(x)=1 \text{ т.к. } x = 0 + 1\cdot x \in L_0$$, а также $$f(y) = 0, \forall y \in L \text{ т.к. } y = y + 0\cdot x \in L_0, \forall y \in L$$ т.е. мы доказали 1. и 2. \\
	
	Должно быть очевидно (после двух доказательств такого вида) что он линеен т.е. $f\in\linear{L_0}{\Real}$. Пусть $z \in L_0 \setminus L$, тогда $z \neq 0$ т.к. $0\in L$ и 
	$$|f(z)| = |t| = 
	|t|\frac{\norm{z}}{\norm{z}} = 
	\frac{|t|\norm{z}}{\norm{y+tx}} =
	\frac{\norm{z}}{\norm{\frac{y}{t} + x}} = 
	\frac{\norm{z}}{\norm{x - (-\frac{y}{t})}}$$
	, т.к. $-\frac{y}{t} \in L$ и $\dist(x, L) = d > 0$, имеем $\norm{x - (-\frac{y}{t})} \geq d$, откуда
	$$|f(z)| = \frac{\norm{z}}{\norm{x - (-\frac{y}{t})}} \leq 
	\frac{1}{d}\norm{z}$$ и доказано $\norm{f} \leq \frac{1}{d}$. \\
	
	Чтобы доказать что $\norm{f} = \frac{1}{d}$, воспользуемся тем что $$\dist(x,L) = \inf_{y \in L}{\norm{x - y}} = d$$, а значит  $\exists\seq{y_n}{n\in\Natural}: \norm{x-y_n} \xrightarrow[n \to \infty]{} d$ и замечаем
	$$1 = f(x) - f(y_n) = f(x - y_n) \leq \norm{f}\norm{x - y_n} \xrightarrow[n \to \infty]{} \norm{f}d$$
	т.е. $\norm{f} \geq \frac{1}{d}$ чем полностью доказывается требуемое 3., остается лишь заметить что по теореме Хана-Банаха оператор продляется с $L_0$ до $X$ с сохранением всех трех свойств.
\end{proof}

\section{Общий вид функционалов в различных пространствах}

% Теорема Рисса об общем виде
\begin{theorem}[\textbf{Рисса об общем виде}]
	Пусть $H$ - г.п.
	\begin{enumerate}
		\item $(u \in H)\wedge(f(v)=(v,u), \forall v\in H) \Rightarrow (f\in H^*) \wedge (\norm{f}_{H^*}=\norm{u}_H)$
		\item $(f\in H^*)\Rightarrow (\exists! u\in H: (f(v)=(v,u),\forall v\in H)\wedge(\norm{f}_{H^*} = \norm{u}_H))$
	\end{enumerate}
\end{theorem}
\begin{proof}
	$ $\newline
	\textbf{1.} $f(v)=(v,u) \Rightarrow f\in L(H,\Real)$ т.к. по аксиомам скалярного произведения 
	$$(\alpha_1 v_1 + \alpha_2 v2, u) = (\alpha_1 v_1, u) + (\alpha_2 v_2, u) = \alpha_1 (v_1, u) + \alpha_2 (v_2, u)$$
	Из неравенства Коши-Буняковского получаем:
	$$|f(v)| = |(v,u)| \leq \norm{v}\norm{u},\forall v\in H \Rightarrow (f\in H^*) \wedge (\norm{f}_{H^*} \leq \norm{u}_H)$$	
	Теперь пусть $u \neq 0$ (случай $u = 0$ тривиален, это нулевой оператор), тогда положим 
	$$\hat{u} = \frac{u}{\norm{u}} \Rightarrow \norm{\hat{u}} = 1$$ 
	тогда $$f(\hat{u}) = (\frac{u}{\norm{u}}, u) = \frac{(u,u)}{\norm{u}} = \norm{u}$$
	и остается заметить что
	$$\norm{u}_H = |f(\hat{u})| \leq \norm{f}_{H^*} \norm{\hat{u}}_H = \norm{f}_{H^*} \Rightarrow \norm{f}_{H^*} = \norm{u}_{H}$$
	\newline
	
	\textbf{2.1.} Пусть $f \in H^*$, если $f = 0$, то $f(v)=(v,0),\forall v \in H$. Если $f \neq 0$, то 
	$$f \neq 0 \Rightarrow \ker{f} \text{ - подпр-во } \neq H \Rightarrow H = \ker{f} \oplus (\ker{f})^\perp$$ 
	Т.к. в этом случае $(\ker{f})^\perp \neq \{0\}$, то можем взять $x_1, x_2 \neq 0, x_1, x_2 \in (\ker{f})^\perp$ и рассмотреть следующий интересный элемент $H$:
	$$x = f(x_2)x_1 - f(x_1)x_2$$
	Понятно что $x \in (\ker{f})^\perp$, в тоже время
	$$f(f(x_2)x_1 - f(x_1)x_2) = f(x_2)f(x_1) - f(x_1)f(x_2) = 0$$ т.е. $x \in \ker{f}$. Как известно $M\cap M^\perp = \{0\}$, так что
	$$f(x_1)x_2 - f(x_2)x_1 = 0, f(x_1) \neq 0, f(x_2) \neq 0 \text{ т.к. не принадлежат ядру }$$
	Мы получили что любые два элемента $(\ker{f})^\perp$ линейно зависимы, а это означает что $\dim{(\ker{f})^\perp} = 1$ - ортогональное дополнение ядра является прямой. \\
	
	\textbf{2.2.} Таким образом $$\exists u_0 \neq 0: (\ker{f})^\perp = \setbuild{u\in H}{\exists t \in \Real: u = t u_0}$$
	и с учетом того что $H = \ker{f} \oplus (\ker{f})^\perp$
	$$\forall v \in H \: \exists!w\in\ker{f} \: \exists!t\in\Real: v = w + tu_0$$ 
	Замечаем
	$$f(v) = f(w) + tf(u_0) = 0 + tf(u_0) \text{ т.к. } w\in\ker{f} $$
	причем т.к. $u_0 \in (\ker{f})^\perp$, то $f(u_0) \neq 0$ и мы получили явное выражение для $t$:
	$$t = \frac{f(v)}{f(u_0)} $$
	
	\textbf{2.3.} Итого $v = w + \frac{f(v)}{f(u_0)} u_0$, посчитаем скалярное произведение:
	$$(v, u_0) = (w, u_0) + \frac{f(v)}{f(u_0)} (u_0, u_0) = f(v) \frac{\norm{u_0}^2}{f(u_0)}$$
	здесь следует напомнить что $(w, u_0) = 0$ т.к. $u_0 \in (\ker{f})^\perp$, а $w \in \ker{f}$
	Теперь остается принять 
	$$u = \frac{u_0}{\norm{u_0}^2} f(u_0) \text{ т.к. тогда } (v, u) = (v, u_0) \frac{f(u_0)}{\norm{u_0}^2} = f(v), \forall v \in H$$
	
	\textbf{2.4.} Когда такой элемент $u$ найден, можно воспользоваться \textbf{1.} и получить $\norm{f}_{H^*} = \norm{u}_H$ т.е. теорема почти доказана, но нам нужна единственность. Пусть 
	$$u_1, u_2 \in H: f(v) = (v, u_1) = (v, u_2), \forall v\in H$$
	, но тогда
	$$(v, u_1 - u_2) = 0,\forall v \in H \Rightarrow (u_1 - u_2, u_1 - u_2) = 0 \Rightarrow u_1 = u_2$$
	
	
\end{proof}
\begin{corollaryth}
	На первый взгляд кажется что эта теорема о том что на гильбертовом пространстве $H$ других линейных ограниченных функционалов кроме как скалярных произведений на некоторых элемент $u\in H$ - нет.
\end{corollaryth}
\begin{corollaryth}
	Тем не менее, её смысл куда глубже, он заключается в том что $H^*$ изометрически изоморфно $H$. Действительно: пусть $F: H^* \to H$ определяемое как $f \xrightarrow{F} u, \text{ где } u \in H: f(v) = (v,u)$ вывод теоремы \textbf{1} о том что это отображение сюръективно, а \textbf{2} о том что оно инъективно, а значит оно взаимно-однозначно. Изометричность видно из соотношений $\norm{f}_{H^*} = \norm{u}_H$ в обоих выводах теоремы, это значит что $\norm{f}_{H^*} = \norm{F(f)}_H$. Линейность $F$ очевидна и теперь можно считать что $H^* \sim H$
\end{corollaryth}

\newpage
\renewcommand{\listtheoremname}{Список теорем и утверждений}
\listoftheorems[ignoreall, show={theorem,corollary}]

\end{document}
