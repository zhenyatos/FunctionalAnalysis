\documentclass[12pt,a4paper]{article}
\usepackage[utf8]{inputenc}
\usepackage[english,russian]{babel}
\usepackage{hyperref}
\usepackage[left=3cm,right=3cm,
top=3cm,bottom=3cm,bindingoffset=0cm]{geometry}
\usepackage{amsthm,amssymb,amsfonts,amsmath}
\hypersetup{
	colorlinks   = true, 
	urlcolor     = blue, 
	linkcolor    = blue, 
	citecolor   = red
}

\theoremstyle{definition}
\newtheorem{exercise}{Задача}

\newenvironment{solution}
{\renewcommand\qedsymbol{$\blacksquare$}\begin{proof}[Решение]}
{\end{proof}}

\newenvironment{note}
{\renewcommand\qedsymbol{}\begin{proof}[Примечание]}
	{\end{proof}}

% Непрерывное вложение
\makeatletter
\newcommand{\rightarrowhead}{\mathrel{%
		\hbox{\let\f@size\sf@size\usefont{U}{lasy}{m}{n}\symbol{41}}}}

\newcommand\arrsubset{\mathrel{\ooalign{$\subset$\cr
			\hidewidth\raise-0.440ex\hbox{$\rightarrowhead\mkern0.5mu$}}}}
% Непрерывное вложение

\newcommand{\Real}{\mathbb{R}}
\newcommand{\Complex}{\mathbb{C}}
\newcommand{\Natural}{\mathbb{N}}
\newcommand{\norm}[1]{\left\lVert#1\right\rVert}
\newcommand{\conj}[1]{\left(#1\right)^*}
\newcommand{\setbuild}[2]{\{#1\:|\:#2\}}
\newcommand{\seq}[2]{\{#1\}_{#2}}
\DeclareMathOperator{\dist}{dist}

\begin{document}
\begin{abstract}
	Везде где не указано иначе, предполагается что $1 < p < \infty$, \\ область $\Omega$ -- односвязная, ограниченная и липшицева
\end{abstract}

\addtocounter{exercise}{1}

\begin{exercise}
	Пусть $\Omega \subset \Real^3$. Пусть $u \in C^\infty (\overline{\Omega})$, причем $(u, v)_{W_2^1 (\Omega)} = 0, \forall v \in \mathring{W_2^1} (\Omega)$. Какому дифференциальному уравнению должна удовлетворять функция $u$?
\end{exercise}
\begin{solution}
	$C_0^\infty (\Omega) \subset \mathring{W_2^1} (\Omega)$ т.к. последнее определяется как замыкание первого в $W_2^1 (\Omega)$, а значит
	\begin{equation}\label{eq:3}
		(u, v)_{W_2^1 (\Omega)} = \int\limits_{\Omega}{\left(u v + \frac{\partial u}{\partial x} \frac{\partial v}{\partial x} + \frac{\partial u}{\partial y} \frac{\partial v}{\partial y} +  \frac{\partial u}{\partial z} \frac{\partial v}{\partial z}\right) d\mu} = 0, \forall v \in C_0^\infty (\Omega)
	\end{equation}
Интегрированием по частям получаем
\begin{equation*}
	\int\limits_{\Omega}{\frac{\partial u}{\partial x} \frac{\partial v}{\partial x} d\mu} = \int\limits_{\partial \Omega}{\frac{\partial u}{\partial x} v n_x dS} - \int\limits_{\Omega}{\frac{\partial^2 u}{\partial x^2} v d\mu} \overset{v \in C_0^\infty (\Omega)}{=} -\int\limits_{\Omega}{\frac{\partial^2 u}{\partial x^2} v d\mu}
\end{equation*}
где $n_x$ -- координата $x$ нормали $n$ к $\partial \Omega$. Аналогично для координат $y$ и $z$, так что \eqref{eq:3} преобразуется в 
\begin{equation*}
	\int\limits_{\Omega}{\left( u - \frac{\partial^2 u}{\partial x^2} - \frac{\partial^2 u}{\partial y^2} - \frac{\partial^2 u}{\partial z^2}\right)v d\mu} = \int\limits_{\Omega}{(u - \Delta u)v d\mu} = 0, \forall v \in C_0^\infty (\Omega)
\end{equation*}
и по лемме Дюбуа-Реймондса получаем
\begin{equation*}
	u - \Delta u = 0 \text{ п.в. в } \Omega
\end{equation*}
однако $u \in C^\infty (\overline{\Omega})$ по условию, поэтому равенство выполняется всюду
\begin{equation*}
	u - \Delta u = 0 \text{ в } \Omega
\end{equation*} 
это и есть дифференциальное уравнение которому удовлетворяет $u$.
\end{solution}
\newpage

\addtocounter{exercise}{4}

\begin{exercise}\label{exr:1}
	Пусть $u \in W_p^1 (\Omega), v \in W_{p^*}^1 (\Omega)$. Показать что $uv \in W_1^1 (\Omega)$ и \\ $\frac{\partial_c (uv)}{\partial x_i} = u\frac{\partial_c v}{\partial x_i} + v\frac{\partial_c u}{\partial x_i}$
\end{exercise}
\begin{solution}
	Из условия следует что $u \in L^p (\Omega)$ и $v \in L^{p^*} (\Omega)$, так что из неравенства Гельдера получаем $uv \in L^1 (\Omega)$. Докажем что 
	\begin{equation}\label{eq:1}
		\frac{\partial_c (u\varphi)}{\partial x_i} = \varphi \frac{\partial_c u}{\partial x_i} + u \frac{\partial \varphi}{\partial x_i}, \forall \varphi \in C^\infty (\Omega)
	\end{equation}
	действительно если $\psi \in C_0^\infty (\Omega)$, то $\varphi \psi \in C_0^\infty (\Omega)$ и 
	\begin{align*}
		\int\limits_{\Omega}{u \frac{\partial (\varphi \psi)}{\partial x_i} dx} &= -\int\limits_{\Omega}{\frac{\partial_c u}{\partial x_i} \varphi \psi dx} \\
		\int\limits_{\Omega}{u \frac{\partial (\varphi \psi)}{\partial x_i} dx} &= \int\limits_{\Omega}{u \psi \frac{\partial \varphi }{\partial x_i} dx} + \int\limits_{\Omega}{u \varphi \frac{\partial \psi}{\partial x_i} dx}
	\end{align*}
	где первое равенство получено по определению Соболевской производной, а второе через классическую формулу производной произведения. Из этих равенств получаем
	\begin{equation*}
		\int\limits_{\Omega}{u \varphi \frac{\partial \psi}{\partial x_i} dx} = - \left( \int\limits_{\Omega}{\frac{\partial_c u}{\partial x_i} \varphi \psi dx} + \int\limits_{\Omega}{u \psi \frac{\partial \varphi }{\partial x_i} dx} \right) = -\int\limits_{\Omega}{\left( \varphi \frac{\partial_c u}{\partial x_i} + u \frac{\partial \varphi}{\partial x_i} \right) \psi dx}
	\end{equation*} 
	ч.т.д.
	Известно что $C^\infty (\Omega)$ всюду плотно в $W_{p^*}^1 (\Omega)$, так что 
	\begin{equation*}
		\exists \{\varphi_k\}_{k\in\Natural} \subset C^\infty (\Omega): \varphi_k \to v \text{ в } W_{p^*}^1 (\Omega)
	\end{equation*}
	отсюда следует что $\varphi_k \to v$ в $L^{p^*} (\Omega)$ и $\frac{\partial \varphi_k}{\partial x_i} \to \frac{\partial_c v}{\partial x_i}$ в $L^{p^*} (\Omega)$, для $i \in \{1, ..., n\}$. Воспользуемся \eqref{eq:1} и получим
	\begin{equation*}
		\frac{\partial_c (u \varphi_k)}{\partial x_i} = \varphi_k \frac{\partial_c u}{\partial x_i} + u \frac{\partial \varphi_k}{\partial x_i} \to v \frac{\partial_c u}{\partial x_i} + u \frac{\partial_c v}{\partial x_i} \text{ в } L^{p^*} (\Omega)
	\end{equation*}
	Значит
	\begin{equation}\label{eq:2}
		\frac{\partial_c (u \varphi_k)}{\partial x_i} \to v \frac{\partial_c u}{\partial x_i} + u \frac{\partial_c v}{\partial x_i} \text{ в } L^1 (\Omega)
	\end{equation}
	т.к. $L^{p^*} (\Omega) \arrsubset L^1 (\Omega)$ и воспользуемся неравенством Гельдера
	\begin{equation*}
		\norm{u\varphi_k - uv}_{L^1 (\Omega)} \leq \norm{u}_{L^p (\Omega)} \norm{\varphi_k - v}_{L^{p^*} (\Omega)} \to 0
	\end{equation*}
	т.е. $u\varphi_k \to uv$ в $L^1 (\Omega)$ и по признаку обобщенной производной с учетом \eqref{eq:2} получаем что
	\begin{equation*}
		\frac{\partial_c (uv)}{\partial x_i} = u\frac{\partial_c v}{\partial x_i} + v\frac{\partial_c u}{\partial x_i}
	\end{equation*}
	и кроме того $\frac{\partial_c (uv)}{\partial x_i} \in L^1 (\Omega)$ как сумма функций из этого пространства, ведь $\frac{\partial_c v}{\partial x_i} \in L^{p^*} (\Omega)$ и $\frac{\partial_c u}{\partial x_i} \in L^p (\Omega)$ известно по условию из того что $v \in W_{p^*}^1 (\Omega)$ и $u \in W_p^1 (\Omega)$ соответственно, а затем вновь применяем Гельдера. Итого $uv \in W_1^1 (\Omega)$, ч.т.д.
\end{solution}
\newpage

\addtocounter{exercise}{8}

\begin{exercise}
	Пусть $\Omega$ -- липшицева область. Пусть $u \in W_p^1 (\Omega), v \in W_{p^*}^1 (\Omega)$ и $\{u_k\}_{k\in\Natural}, \{v_k\}_{k\in\Natural} \subset C^\infty (\overline{\Omega})$ таковы что $u_k \to u$ в $W_p^1 (\Omega)$, $v_k \to v$ в $W_{p^*}^1 (\Omega)$. Показать что $u_k v_k \to u v$ в $W_1^1 (\Omega)$
\end{exercise}
\begin{solution}
	Воспользуемся \hyperref[exr:1]{задачей 7}, получив что $uv \in W_1^1 (\Omega)$, далее заметим
	\begin{multline*}
		\norm{u_k v_k - u v}_{1, 1, \Omega} = \norm{u_k v_k - u_k v + u_k v - u v}_{1, 1, \Omega} \leq \\ \leq \norm{u_k (v_k - v)}_{1, 1, \Omega} + \norm{(u_k - u)v}_{1, 1, \Omega} \leq \cdots
	\end{multline*}
	Теперь необходимо получить аналог неравенства Гельдера для пространств $W_p^1 (\Omega)$, один из возможных путей (грубая оценка):
	\begin{multline*}
		\norm{uv}_{1, 1, \Omega} = \norm{uv}_{L^1 (\Omega)} + \sum\limits_{i=1}^{n}{\norm{\frac{\partial_c (uv)}{\partial x_i}}_{L^1 (\Omega)}} = \norm{uv}_{L^1 (\Omega)} + \sum\limits_{i=1}^{n}{\norm{ v \frac{\partial_c u}{\partial x_i} + u \frac{\partial_c v}{\partial x_i} }_{L^1 (\Omega)}} \leq \\ \leq \norm{uv}_{L^1 (\Omega)} + \sum\limits_{i=1}^{n}{\norm{v \frac{\partial_c u}{\partial x_i}}_{L^1 (\Omega)}} + \sum\limits_{i=1}^{n}{\norm{u \frac{\partial_c v}{\partial x_i}}_{L^1 (\Omega)}} \leq \\
		\leq \norm{u}_{L^p (\Omega)}\norm{v}_{L^{p^*} (\Omega)} + \sum\limits_{i=1}^{n}{\left( \norm{\frac{\partial_c u}{\partial x_i}}_{L^p (\Omega)} \norm{v}_{L^{p^*} (\Omega)} \right)} + \sum\limits_{i=1}^{n}{\left(\norm{\frac{\partial_c v}{\partial x_i}}_{L^{p^*} (\Omega)} \norm{u}_{L^p (\Omega)} \right)} \leq \\ \leq (2n+1) \norm{u}_{1, p, \Omega} \norm{v}_{1, p^*, \Omega}
	\end{multline*}
	Итого
	\begin{equation}
		\norm{uv}_{1, 1, \Omega} \leq (2n+1)\norm{u}_{1, p, \Omega}\norm{v}_{1, p^*, \Omega}
	\end{equation}
	Т.к. $u_k \to u$ в $W_p^1 (\Omega)$, то $\exists M \geq 0: \norm{u_k}_{1, p, \Omega} \leq M, \forall k \in \Natural$ и воспользовавшись этим получаем
	\begin{multline*}
		\cdots \leq (2n+1)\norm{u_k}_{1, p, \Omega}\norm{v_k - v}_{1, p^*, \Omega} + (2n+1)\norm{u_k - u}_{1, p, \Omega}\norm{v}_{1, p^*, \Omega} \leq \\ \leq
		(2n+1) \left( M \norm{v_k - v}_{1, p^*, \Omega} + \norm{u_k - u}_{1, p, \Omega} \norm{v}_{1, p^*, \Omega} \right) \to 0
	\end{multline*}
	т.к. $\norm{v_k - v}_{1, p^*, \Omega} \to 0$ и $\norm{u_k - u}_{1, p, \Omega} \to 0$ т.е. действительно $\norm{u_k v_k - uv}_{1, 1, \Omega} \to 0$, ч.т.д.
\end{solution}
\newpage
	
\end{document}