\documentclass[12pt,a4paper]{article}
\usepackage[utf8]{inputenc}
\usepackage[english,russian]{babel}
\usepackage{hyperref}
\usepackage[left=3cm,right=3cm,
top=3cm,bottom=3cm,bindingoffset=0cm]{geometry}
\usepackage{amsthm,amssymb,amsfonts,amsmath}
\hypersetup{
	colorlinks   = true, 
	urlcolor     = blue, 
	linkcolor    = blue, 
	citecolor   = red
}

\theoremstyle{definition}
\newtheorem{exercise}{Задача}

\newenvironment{solution}
{\renewcommand\qedsymbol{$\blacksquare$}\begin{proof}[Решение]}
{\end{proof}}

\newenvironment{note}
{\renewcommand\qedsymbol{}\begin{proof}[Примечание]}
	{\end{proof}}

% Непрерывное вложение
\makeatletter
\newcommand{\rightarrowhead}{\mathrel{%
		\hbox{\let\f@size\sf@size\usefont{U}{lasy}{m}{n}\symbol{41}}}}

\newcommand\arrsubset{\mathrel{\ooalign{$\subset$\cr
			\hidewidth\raise-0.440ex\hbox{$\rightarrowhead\mkern0.5mu$}}}}
% Непрерывное вложение

\newcommand{\Real}{\mathbb{R}}
\newcommand{\Complex}{\mathbb{C}}
\newcommand{\Natural}{\mathbb{N}}
\newcommand{\norm}[1]{\left\lVert#1\right\rVert}
\newcommand{\conj}[1]{\left(#1\right)^*}
\newcommand{\setbuild}[2]{\{#1\:|\:#2\}}
\newcommand{\seq}[2]{\{#1\}_{#2}}
\DeclareMathOperator{\dist}{dist}

\begin{document}
\begin{abstract}
	Везде где не указано иначе, предполагается что $1 < p < \infty$, \\ область $\Omega$ -- односвязная, ограниченная и липшицева
\end{abstract}
\addtocounter{exercise}{6}

\begin{exercise}
	Пусть $u \in W_p^1 (\Omega), v \in W_{p^*}^1 (\Omega)$. Показать что $uv \in W_1^1 (\Omega)$ и \\ $\frac{\partial_c (uv)}{\partial x_i} = u\frac{\partial_c v}{\partial x_i} + v\frac{\partial_c u}{\partial x_i}$
\end{exercise}
\begin{solution}
	Из условия следует что $u \in L^p (\Omega)$ и $v \in L^{p^*} (\Omega)$, так что из неравенства Гельдера получаем $uv \in L^1 (\Omega)$. Докажем что 
	\begin{equation}\label{eq:1}
		\frac{\partial_c (u\varphi)}{\partial x_i} = \varphi \frac{\partial_c u}{\partial x_i} + u \frac{\partial \varphi}{\partial x_i}, \forall \varphi \in C^\infty (\Omega)
	\end{equation}
	действительно если $\psi \in C_0^\infty (\Omega)$, то $\varphi \psi \in C_0^\infty (\Omega)$ и 
	\begin{align*}
		\int\limits_{\Omega}{u \frac{\partial (\varphi \psi)}{\partial x_i} dx} &= -\int\limits_{\Omega}{\frac{\partial_c u}{\partial x_i} \varphi \psi dx} \\
		\int\limits_{\Omega}{u \frac{\partial (\varphi \psi)}{\partial x_i} dx} &= \int\limits_{\Omega}{u \psi \frac{\partial \varphi }{\partial x_i} dx} + \int\limits_{\Omega}{u \varphi \frac{\partial \psi}{\partial x_i} dx}
	\end{align*}
	где первое равенство получено по определению Соболевской производной, а второе через классическую формулу производной произведения. Из этих равенств получаем
	\begin{equation*}
		\int\limits_{\Omega}{u \varphi \frac{\partial \psi}{\partial x_i} dx} = - \left( \int\limits_{\Omega}{\frac{\partial_c u}{\partial x_i} \varphi \psi dx} + \int\limits_{\Omega}{u \psi \frac{\partial \varphi }{\partial x_i} dx} \right) = -\int\limits_{\Omega}{\left( \varphi \frac{\partial_c u}{\partial x_i} + u \frac{\partial \varphi}{\partial x_i} \right) \psi dx}
	\end{equation*} 
	ч.т.д.
	Известно что $C^\infty (\Omega)$ всюду плотно в $W_{p^*}^1 (\Omega)$, так что 
	\begin{equation*}
		\exists \{\varphi_k\}_{k\in\Natural} \subset C^\infty (\Omega): \varphi_k \to v \text{ в } W_{p^*}^1 (\Omega)
	\end{equation*}
	отсюда следует что $\varphi_k \to v$ в $L^{p^*} (\Omega)$ и $\frac{\partial \varphi_k}{\partial x_i} \to \frac{\partial_c v}{\partial x_i}$ в $L^{p^*} (\Omega)$, для $i \in \{1, ..., n\}$. Воспользуемся \eqref{eq:1} и получим
	\begin{equation*}
		\frac{\partial_c (u \varphi_k)}{\partial x_i} = \varphi_k \frac{\partial_c u}{\partial x_i} + u \frac{\partial \varphi_k}{\partial x_i} \to v \frac{\partial_c u}{\partial x_i} + u \frac{\partial_c v}{\partial x_i} \text{ в } L^{p^*} (\Omega)
	\end{equation*}
	Значит
	\begin{equation}\label{eq:2}
		\frac{\partial_c (u \varphi_k)}{\partial x_i} \to v \frac{\partial_c u}{\partial x_i} + u \frac{\partial_c v}{\partial x_i} \text{ в } L^1 (\Omega)
	\end{equation}
	т.к. $L^{p^*} (\Omega) \arrsubset L^1 (\Omega)$ и воспользуемся неравенством Гельдера
	\begin{equation*}
		\norm{u\varphi_k - uv}_{L^1 (\Omega)} \leq \norm{u}_{L^p (\Omega)} \norm{\varphi_k - v}_{L^{p^*} (\Omega)} \to 0
	\end{equation*}
	т.е. $u\varphi_k \to uv$ в $L^1 (\Omega)$ и по признаку обобщенной производной с учетом \eqref{eq:2} получаем что
	\begin{equation*}
		\frac{\partial_c (uv)}{\partial x_i} = u\frac{\partial_c v}{\partial x_i} + v\frac{\partial_c u}{\partial x_i}
	\end{equation*}
	и кроме того $\frac{\partial_c (uv)}{\partial x_i} \in L^1 (\Omega)$ как сумма функций из этого пространства, ведь $\frac{\partial_c v}{\partial x_i} \in L^{p^*} (\Omega)$ и $\frac{\partial_c u}{\partial x_i} \in L^p (\Omega)$ известно по условию из того что $v \in W_{p^*}^1 (\Omega)$ и $u \in W_p^1 (\Omega)$ соответственно, а затем вновь применяем Гельдера. Итого $uv \in W_1^1 (\Omega)$, ч.т.д.
\end{solution}

\newpage
	
\end{document}