\documentclass[12pt,a4paper]{article}
\usepackage[utf8]{inputenc}
\usepackage[english,russian]{babel}
\usepackage{hyperref}
\usepackage{amsthm,amssymb,amsfonts,amsmath}

\theoremstyle{definition}
\newtheorem{exercise}{Задача}[section]

\newenvironment{solution}
{\renewcommand\qedsymbol{$\blacksquare$}\begin{proof}[Решение]}
{\end{proof}}

\newenvironment{note}
{\renewcommand\qedsymbol{}\begin{proof}[Примечание]}
	{\end{proof}}

\newcommand{\Real}{\mathbb{R}}
\newcommand{\Cmplx}{\mathbb{C}}
\newcommand{\norm}[1]{\left\lVert#1\right\rVert}

\begin{document}
	\section{Линейные операторы}
	
	\begin{exercise}
		Пусть $X,Y$ - банаховы пространства, $A\in B(X,Y)$ Всегда ли равенство $\norm{x}_1 = \norm{x} + \norm{Ax}$ задает в $X$ норму? Если да, будет ли $(X, \norm{\cdot}_1)$ - банаховым пространством.
	\end{exercise}
	\begin{solution}
		Проверим удовлетворяет ли $\norm{\cdot}_1$ аксиомам нормы:
		\begin{enumerate}
			\item $\norm{x_1 + x_2}_1 = \norm{x_1 + x_2} + \norm{A(x_1 + x_2)} \leq \norm{x_1} + \norm{x_2} + \norm{Ax_1 + Ax_2} \leq \newline
			\leq \norm{x_1} + \norm{x_2} + \norm{Ax_1} + \norm{Ax_2}
			\leq \norm{x_1}_1 + \norm{x_2}_1$
			
			\item $\norm{\alpha x}_1 = \norm{\alpha x} + \norm{A(\alpha x)} = |\alpha| \norm{x} + \norm{\alpha Ax} = |\alpha| (\norm{x} + \norm{Ax}) = \newline = |\alpha|\norm{x}_1$
			
			\item $\norm{x}_1 = 0 \Leftrightarrow \norm{x} + \norm{Ax} = 0 \Leftrightarrow \norm{x} = 0 \wedge \norm{Ax} = 0 \Leftrightarrow x = 0$, где предпоследний переход связан с неотрицательностью нормы, а последний с тем что $A0=0$ для линейного оператора
		\end{enumerate}
		$(X, \norm{\cdot}_1)$ будет банаховым пространством т.к. $\norm{\cdot}_1 \sim \norm{\cdot}$. Действительно $\norm{x}_1 = \norm{x} + \norm{Ax} \leq \norm{x} + \norm{A}\norm{x} = (1 + \norm{A})\norm{x}$ и очевидно $\norm{x}_1 \geq \norm{x}$ так что
		$\forall x\in X : \norm{x} \leq \norm{x}_1 \leq (1 + \norm{A})\norm{x}$.
	\end{solution}

	\begin{note}
		В решении не использовал банаховость $X$, только нормированность, банаховость $Y$ вероятно нужна для того чтобы оператор $A$ можно было считать заданным на всем пространстве $X$, с учетом теоремы о продолжении оператора. Иначе указанное равенство вообще будет определено не для всех $x$ и разумеется это не норма.
	\end{note}

	\newpage
	
\end{document}