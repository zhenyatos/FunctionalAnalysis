\documentclass[12pt,a4paper]{article}
\usepackage[utf8]{inputenc}
\usepackage[english,russian]{babel}
\usepackage{hyperref}
\usepackage[left=3cm,right=3cm,
top=3cm,bottom=3cm,bindingoffset=0cm]{geometry}
\usepackage{amsthm,amssymb,amsfonts,amsmath}

\theoremstyle{definition}
\newtheorem{exercise}{Задача}[section]

\newenvironment{solution}
{\renewcommand\qedsymbol{$\blacksquare$}\begin{proof}[Решение]}
{\end{proof}}

\newenvironment{note}
{\renewcommand\qedsymbol{}\begin{proof}[Примечание]}
	{\end{proof}}

\newcommand{\Real}{\mathbb{R}}
\newcommand{\Complex}{\mathbb{C}}
\newcommand{\Natural}{\mathbb{N}}
\newcommand{\norm}[1]{\left\lVert#1\right\rVert}
\newcommand{\conj}[1]{\left(#1\right)^*}
\newcommand{\setbuild}[2]{\{#1\:|\:#2\}}
\newcommand{\seq}[2]{\{#1\}_{#2}}
\DeclareMathOperator{\dist}{dist}

\begin{document}
	\section{Теория меры}
	
	\begin{exercise}
		Применить теорему Егорова к последовательности функций $f_n(x) = x^n$ на отрезке $[0,1]$.
	\end{exercise}
	\begin{solution}
		Проверим каждое из условий теоремы Егорова, $f(x) = 0$:
		\begin{enumerate}
			\item $[0,1]$ - измеримо как замкнутый параллелепипед в $\Real$, причем \newline $\mu([0,1]) = v([0,1]) = 1-0 = 1 < +\infty$
			
			\item $\forall n \in \Natural: f_n \text{ - непрерывна на } [0,1]$, а мы уже показали что $[0,1]$ - измеримо, так что $\forall n \in \Natural: f_n \text{ - измерима на } [0,1]$
			
			\item $\forall x \in [0, 1]: (|f_n(x)| = x^n \leq 1 < +\infty) \wedge (|f(x)| = 0 < +\infty)$ т.е. $f_n, f$ - конечны на $[0,1]$
			
			\item Наконец, заметим что $\forall x \in [0,1): f_n(x) = x^n \to 0 = f(x)$ при $n \to \infty$ и $\mu(\{1\}) = 0$, а $[0,1] \setminus [0,1) = \{1\}$ т.е. $f_n \to f$ почти всюду на $[0,1]$
		\end{enumerate}
	
		Условия теоремы выполнены и мы получаем следующий вывод: $$\forall \varepsilon > 0 \: \exists E_{\varepsilon} \text{ - измеримое}: (E_{\varepsilon} \subset [0,1]) \wedge (\mu([0,1]\setminus E_{\varepsilon}) < \varepsilon) \wedge (x^n \rightrightarrows 0 \text{ на } E_{\varepsilon})$$ т.е. существует сколь угодно "близкое" по мере к $E$ измеримое подмножество на котором сходимость равномерная. 
	\end{solution}
	\begin{note}
		По существу результат довольно смешной т.к. в данном случае такое подмножество можно предъявить непосредственно, к примеру $[0, 1 - \frac{\delta}{2}], \delta = \min\{\varepsilon, 1\}$. Тем не менее задача полезна методически для понимания условий теоремы Егорова.
	\end{note}
	

	\section{Линейные операторы}
	
	\begin{exercise}
		Пусть $X,Y$ - банаховы пространства, $A\in B(X,Y)$ Всегда ли равенство $\norm{x}_1 = \norm{x} + \norm{Ax}$ задает в $X$ норму? Если да, будет ли $(X, \norm{\cdot}_1)$ - банаховым пространством.
	\end{exercise}
	\begin{solution}
		Проверим удовлетворяет ли $\norm{\cdot}_1$ аксиомам нормы:
		\begin{enumerate}
			\item $\norm{x_1 + x_2}_1 = \norm{x_1 + x_2} + \norm{A(x_1 + x_2)} \leq \norm{x_1} + \norm{x_2} + \norm{Ax_1 + Ax_2} \leq \newline
			\leq \norm{x_1} + \norm{x_2} + \norm{Ax_1} + \norm{Ax_2}
			\leq \norm{x_1}_1 + \norm{x_2}_1$
			
			\item $\norm{\alpha x}_1 = \norm{\alpha x} + \norm{A(\alpha x)} = |\alpha| \norm{x} + \norm{\alpha Ax} = |\alpha| (\norm{x} + \norm{Ax}) = \newline = |\alpha|\norm{x}_1$
			
			\item $\norm{x}_1 = 0 \Leftrightarrow \norm{x} + \norm{Ax} = 0 \Leftrightarrow \norm{x} = 0 \wedge \norm{Ax} = 0 \Leftrightarrow x = 0$, где предпоследний переход связан с неотрицательностью нормы, а последний с тем что $A0=0$ для линейного оператора
		\end{enumerate}
		$(X, \norm{\cdot}_1)$ будет банаховым пространством т.к. $\norm{\cdot}_1 \sim \norm{\cdot}$. Действительно \\ $\norm{x}_1 = \norm{x} + \norm{Ax} \leq \norm{x} + \norm{A}\norm{x} = (1 + \norm{A})\norm{x}$ и очевидно $\norm{x}_1 \geq \norm{x}$ так что
		$\forall x\in X : \norm{x} \leq \norm{x}_1 \leq (1 + \norm{A})\norm{x}$.
	\end{solution}

	\begin{note}
		В решении не использовал банаховость $X$, только нормированность, банаховость $Y$ вероятно нужна для того чтобы оператор $A$ можно было считать заданным на всем пространстве $X$, с учетом теоремы о продолжении оператора. Иначе указанное равенство вообще будет определено не для всех $x$ и разумеется это не норма.
	\end{note}

	\section{Линейные функционалы}
	
	\begin{exercise}
		Пусть $X$ - н.п., $M \subset X$ и $M^\perp = \setbuild{f \in X^*}{f(x) = 0, \forall x \in M}$. Доказать что $M^\perp$ - подпространство в $X^*$. \\ Пусть $M$ - подпространство в $X$, доказать что $M = \setbuild{x\in X}{f(x) = 0, \forall f \in M^\perp}$
	\end{exercise}
	\begin{solution}
		Пусть $M^\perp = \setbuild{f \in X^*}{f(x) = 0, \forall x \in M}$, докажем что $M^\perp$ - подпространство в $X$:
		\begin{enumerate}
			\item $f_1, f_2 \in M^\perp \Leftrightarrow (f_1, f_2 \in X^*) \wedge (f_1 (x) = f_2 (x) = 0, \forall x \in M) \Rightarrow \\
			\Rightarrow (\alpha_1 f_1 + \alpha_2 f_2 \in X^*) \wedge ((\alpha_1 f_1 + \alpha_2 f_2 )(x) = \alpha_1 f_1 (x) + \alpha_2 f_2 (x) = 0, \forall x \in M) \Leftrightarrow \\
			\Leftrightarrow \alpha_1 f_1 + \alpha_2 f_2 \in M^\perp$
			
			\item $(\seq{f_n}{n\in \Natural} \subset M^\perp: f_n \to f, f\in X^*) \Rightarrow \\
			\Rightarrow (\norm{f(x)} = \norm{f_n(x) - f(x)} = \norm{(f_n - f)(x)} \leq \norm{f_n - f}\norm{x} \to 0, \\ \forall x \in M^\perp) \Rightarrow (f(x) = 0, \forall x \in M^\perp)$
		\end{enumerate}
		Доказано что $M^\perp$ - замкнутое линейное многообразие т.е. действительно подпространство.
		\newline \newline
		Пусть $M$ - подпространство в $X$, определим $$M_0 = \setbuild{x\in X}{f(x) = 0, \forall f \in M^\perp}$$
		и докажем что $M = M_0$. 
		\begin{enumerate}
			\item Пусть $x \in M$, тогда $\forall f \in M^\perp: f(x) = 0$ по определению $M^\perp$ т.е. $x \in M_0$. 
			\item Пусть $x \notin M$, тогда $\dist(x, M) \neq 0$ т.к. $M$ - замкнуто по определиню подпространства. Значит по следствию 1.3 теоремы Хана-Банаха 
			$$\exists f \in X^*: (f(x) = 1 \neq 0) \wedge (f(y) = 0, \forall y \in M)$$
			т.е. $f \in M^\perp$, но $f(x) \neq 0$, а значит $x \notin M_0$
		\end{enumerate}
	\end{solution}

	\newpage
	
\end{document}