\documentclass[12pt,a4paper]{article}
\usepackage[utf8]{inputenc}
\usepackage[english,russian]{babel}
\usepackage{hyperref}
\usepackage[left=3cm,right=3cm,
top=3cm,bottom=3cm,bindingoffset=0cm]{geometry}
\usepackage{amsthm,amssymb,amsfonts,amsmath}

\hypersetup{
	colorlinks   = true, 
	urlcolor     = blue, 
	linkcolor    = blue, 
	citecolor   = red
}

\theoremstyle{definition}
\newtheorem{exercise}{Задача}[section]

\newenvironment{solution}
{\renewcommand\qedsymbol{$\blacksquare$}\begin{proof}[Решение]}
{\end{proof}}

\newenvironment{note}
{\renewcommand\qedsymbol{}\begin{proof}[Примечание]}
	{\end{proof}}

\newcommand{\Real}{\mathbb{R}}
\newcommand{\Complex}{\mathbb{C}}
\newcommand{\Natural}{\mathbb{N}}
\newcommand{\norm}[1]{\left\lVert#1\right\rVert}
\newcommand{\conj}[1]{\left(#1\right)^*}
\newcommand{\setbuild}[2]{\{#1\:|\:#2\}}
\newcommand{\seq}[2]{\{#1\}_{#2}}
\newcommand{\bounded}[2]{\textrm{B}(#1, #2)}
\newcommand{\linear}[2]{\textrm{L}(#1, #2)}
\DeclareMathOperator{\dist}{dist}
\DeclareMathOperator{\Lin}{Lin}

\begin{document}
	\section{Линейные функционалы}
	
	\begin{exercise}
		Привести пример функционала $f_0 \in \conj{l^\infty}$, такого что он не определяется элементом из $l^1$, то есть не существует такого $\xi \in l^1$, что \\ $f_0 (x) = \sum\limits_{i=1}^{\infty}{\xi_i x_i}, \forall x \in l^{\infty}$
	\end{exercise}
	\begin{solution} Введем ряд обозначений:
		\begin{align*}
			e_1 &= (1, 0, ..., 0, 0, 0, ...) 	\\
			e_2 &= (0, 1, ..., 0, 0, 0, ...) 	\\
				&\cdots						\\
			e_k &= (0, 0, ..., 0, 1, 0, ...)	\\
				&\cdots 
		\end{align*}
		Т.е. для $k \in \Natural$ в $e_k$ - на $k$-ой позиции 1, на остальных 0. Нетрудно видеть что $\forall k \in \Natural: (e_k \in l^1) \wedge (e_k \in l^\infty)$, это нам пригодится в дальнейшем.
		
		\begin{align*}
			e_1^\prime 	&= \left(1, 1, ..., 1, 1, 1, ...\right) \\
			e_2^\prime 	&= \left(0, 1, ..., 1, 1, 1, ...\right) 			\\
						&\cdots 				\\
			e_k^\prime 	&= \left(0, 0, ..., 1, 1, 1, ...,\right) \\
						&\cdots 
		\end{align*}
		Т.е. 0 до $(k-1)$-ой позиции включительно, а после 1. Опять же совершенно очевидно что $e_k^\prime \in l^\infty$, причем $\norm{e_k^\prime}_{l^\infty} = 1$. В тоже время $e_k^\prime \notin l^1$ т.к. ряд имеющий вид $\sum\limits_{i=1}^\infty{1}$ очевидно расходится (обращаем внимание на индекс - это ряд из констант).
		
		Заметим связь введенных векторов (работаем в пространстве $l^\infty$ которому все они принадлежат):
		\begin{align}\label{eq:2}
		\begin{split}
			e_1 &= e_1^\prime - e_2^\prime \\
			e_2 &= e_2^\prime - e_3^\prime \\
					   &\cdots \\
			e_k &= e_{k}^\prime - e_{k+1}^\prime \\
					   &\cdots
		\end{split}
		\end{align}
		
		Теперь мы можем приступить к построению функционала $f$ на линейном многообразии $L = \Lin\{e_1^\prime, ..., e_k^\prime, ...\} \subset l^\infty$. Причем наличие связи между $e_k$ и $e_k^\prime$ говорит о том что $e_k \in L, \forall k \in \Natural$. Достаточно определить $f$ на $e_k^\prime, \forall k \in \Natural$:
		\begin{equation}\label{eq:3}
			f(e_k^\prime) = 1 ,
		\end{equation}
		А на остальных элементах $L$
		$$f(x) = f\left(\sum_{i=1}^{n}{\alpha_i e_i^\prime}\right) = \sum_{i=1}^{n}{\alpha_i f(e_i^\prime)} = \sum_{i=1}^{n}{\alpha_i}, \forall x \in L$$
		т.е. $f \in \linear{l^\infty}{\Real}, D(f) = L$ по построению. Докажем ограниченность:
		\begin{equation}\label{eq:1}
			|f(x)| = \left|\sum_{i=1}^{n}{\alpha_i}\right| \leq \norm{x}_{l^\infty}, \forall x \in L
		\end{equation}
		где 
		\begin{align*}
			x &= \sum\limits_{i=1}^{n}{\alpha_i e_i^\prime} =
			\left(\alpha_1, \alpha_1 + \alpha_2, ..., \sum_{i=1}^{k}{\alpha_i} ..., \sum_{i=1}^{n}{\alpha_i}, \sum_{i=1}^{n}{\alpha_i}, ... \right) \\
			&\Rightarrow \\
			\norm{x}_{l^\infty} &= \sup_{k \in \Natural}{|x_k|} = \max_{k = 1, ..., n}{\left| \sum_{i=1}^{k}{\alpha_i} \right|}
		\end{align*}
		Последнее верно т.к. $x_k$ при $k \geq n$ принимает одно и тоже значение. Теперь \eqref{eq:1} очевидно и можно заключить $f \in \bounded{l^\infty}{\Real}, D(f) = L, \norm{f}_{\bounded{l^\infty}{\Real}} \leq 1$. 
		
		Воспользуемся теоремой Хана-Банаха и получим $F \in \conj{l^\infty}$ - продолжение $f$. Это и есть искомый функционал т.к. пусть $F(x) = \sum\limits_{i=1}^{\infty}{\xi_i x_i}, \forall x \in l^\infty$ для некоторого $\xi \in l^1$. Но тогда:
		\begin{align*}
			F(e_k) &= f(e_k) \overset{\eqref{eq:2}}{=} f( e_{k}^\prime - e_{k+1}^\prime) = f(e_{k} ^\prime) - f(e_{k+1}^\prime)  \overset{\eqref{eq:3}}{=} 1 - 1 = 0 \\
			F(e_k) &= \sum\limits_{i=1}^{\infty}{\xi_i (e_k)_{i}} = \xi_k
		\end{align*}
		Таким образом $\xi_k = 0, \forall k \in \Natural$ и $F$ - нулевой функционал, но это не так по построению. 
	\end{solution}

	\newpage
	
\end{document}