\documentclass[12pt,a4paper]{article}
\usepackage[utf8]{inputenc}
\usepackage[english,russian]{babel}
\usepackage{hyperref}
\hypersetup{
	colorlinks   = true, 
	urlcolor     = blue, 
	linkcolor    = blue, 
	citecolor   = red
}
\usepackage{amsthm,amssymb,amsfonts,amsmath}
\usepackage{physics}
\usepackage{xcolor}
\usepackage[left=3cm,right=3cm,
top=3cm,bottom=3cm,bindingoffset=0cm]{geometry}
\usepackage{tcolorbox}
\usepackage{physics}

\tcbuselibrary{theorems}
\newtcbtheorem
[number within=section, 
list inside={def}]
{definition}
{Определение}
{
	colback=red!3,
	colframe=red!20!white,
	coltitle=black,
	fonttitle=\bfseries,
	sharp corners,
}
{def}

\tcbuselibrary{theorems}
\newtcbtheorem
[number within=section]
{proposition}
{Утверждение}
{
	colback=purple!3,
	colframe=purple!25!white,
	coltitle=black,
	fonttitle=\bfseries,
}
{prop}

\tcbuselibrary{theorems}
\newtcbtheorem
[number within=section]
{theorem}
{Теорема}
{
	colback=blue!3,
	colframe=blue!20!white,
	coltitle=black,
	fonttitle=\bfseries,
}
{th}

\tcbuselibrary{theorems}
\newtcbtheorem
[number within=section]
{example}
{Пример}
{
	colback=orange!3,
	colframe=orange!20!white,
	coltitle=black,
	fonttitle=\bfseries,
	sharp corners,
}
{exmpl}

\tcbuselibrary{theorems}
\newtcbtheorem
[number within=section]
{lemma}
{Лемма}
{
	colback=gray!3,
	colframe=gray!20!white,
	coltitle=black,
	fonttitle=\bfseries,
}
{lemm}

% Для удобства
\newcommand{\intset}[1]{\int\limits_{#1}}
\newcommand{\Real}{\mathbb{R}}
\newcommand{\Natural}{\mathbb{N}}
\newcommand{\ssubset}{\subset \subset}
\newcommand{\Dto}{\overset{\mathcal{D}}{\to}}
\newcommand{\nnorm}[1]{{\left\vert\kern-0.25ex\left\vert\kern-0.25ex\left\vert #1 
		\right\vert\kern-0.25ex\right\vert\kern-0.25ex\right\vert}}
\DeclareMathOperator\supp{supp}
\DeclareMathOperator\dist{dist}
\DeclareMathOperator\diam{diam}

\title{Соболевские пространства}

\begin{document}
	
\maketitle
\begin{abstract}
	Собраны основные понятия и утверждения курса, без доказательств (возможно они появятся в будущем)
\end{abstract}

\section{Усреднение функций}
	
\subsection{Основные понятия}

Пусть в $\Real^n$ задана норма 
$$\norm{x} = \sqrt{\sum_{i=1}^{n}{x_i^2}}$$ 
Определим $\omega: \Real^n \to \Real$ как
\begin{equation*}
	\omega (x) = 
		\begin{cases}
		c e^{\frac{1}{\norm{x}^2 - 1}}, & x \in B(0, 1) \\
		0, & x \notin B(0, 1)
		\end{cases}
\end{equation*}
где 
$$c  = \frac{1}{\intset{B(0,1)}{e^{\frac{1}{\norm{x}^2 - 1}}dx}}$$
так что $\intset{B(0,1)}{\omega(x)dx} = 1$ и кроме того по определению $\omega \in C(\Real^n)$. \\ Пусть $\omega_\rho (x) = \rho^{-n} \omega \left(\frac{x}{\rho}\right)$, тогда свойства указанные выше сохраняются.

\begin{definition}{Усреднение функции}{def:1}
	Пусть $\rho > 0$, усреднением функции $u \in L^{1}(\Omega)$ называется 
	$$u_\rho (x) = \intset{\Omega}{\omega_\rho (x - y) u(y) dy}, \forall x \in \Real^n$$
	Важное свойство $u_\rho \in C^{\infty}(\Real^n), \forall \rho > 0$
\end{definition}

\begin{proposition}{Вложенность $L^p$}{prop:1}
	Пусть $\Omega \subset \Real^n$ -- открытое ограниченное, тогда
	\begin{enumerate}
		\item $L^{\infty}(\Omega) \subset \underset{1 < p < \infty}{L^{p}(\Omega)} \subset L^{1}(\Omega)$
		\item $L^{p}(\Omega) \subset L^{q}(\Omega), 1 \leq q < p \leq \infty$
	\end{enumerate}
\end{proposition}

\begin{proposition}{О норме усреднения}{prop:2}
	Пусть $u \in L^{p}(\Omega), 1 \leq p \leq \infty$, тогда 
	$$\norm{u_\rho}_{L^{p}(\Omega)} \leq \norm{u}_{L^{p}(\Omega)}$$
\end{proposition}

\begin{proposition}{О сходимости усреднений в $C$}{prop:3}
	Пусть $u \in C(\Omega)$, $\underset{\text{компакт}}{K} \subset \Omega$, тогда $u_\rho \underset{\rho \to 0}{\to} u$ в $C(K)$
\end{proposition}
Из последнего утверждения следует что $u_\rho (x) \underset{\rho \to 0}{\to} u (x), \forall x \in \Omega$.

\begin{theorem}{О сходимости усреднений в $L^{p}$}{th:1}
	Пусть $u \in L^{p}(\Omega), 1 \leq p < \infty$, тогда $u_\rho \underset{\rho \to 0}{\to} u$ в $L^{p} (\Omega)$
\end{theorem}

\subsection{Свойства функций из $L^p$ связанные с усреднением}

Рассмотрим множества плотные в пространствах Лебега.
\begin{proposition}{}{prop:4}
	$C^{\infty} (\overline{\Omega})$ плотно в $L^p (\Omega), 1 \leq p < \infty$
\end{proposition}

\begin{proposition}{}{prop:5}
	$C^{\infty}_0 (\Omega)$ плотно в $L^p (\Omega), 1 \leq p < \infty$
\end{proposition}

\begin{definition}{Вхождение с замыканием}{def:2}
	$\Omega^{'} \ssubset \Omega$ если $\Omega^{'}$ -- предкомпакт и $\overline{\Omega^{'}} \subset \Omega$
\end{definition}

\begin{definition}{Локально суммируемые функции}{def:3}
	$u \in L^{p}_{\text{loc}} (\Omega)$ -- $u$ локально суммируема с показателем $p$, если \\ $u \in L^{p} (\Omega^{'}), \forall \Omega^{'} \ssubset \Omega$
\end{definition}
Заметим что на ограниченной области локально суммируемые функции ведут себя как угодно на границе.

\begin{example}{}{exmpl:1}
	$f \equiv c$, $f \notin L^1 (\Real)$, но $f \in L^1_{\text{loc}} (\Real)$
\end{example}

\begin{lemma}{Дюбуа-Реймонда}{}
	Пусть $u \in L^1_{\text{loc}} (\Omega)$ и $\intset{\Omega}{u\varphi dx}, \forall \varphi \in C^{\infty}_0 (\Omega)$, тогда $u = 0$ п.в. на $\Omega$ [Заметим что класс для $u$ очень широкий, а для $\varphi$ очень узкий]
\end{lemma}

\subsection{Критерий компактности в $L^p$}

Пусть $\Omega$ -- ограниченная область, $F \subset L^p (\Omega), 1 \leq p < \infty$, по умолчанию $f \in F$ продолжаем 0 на $\Real^n$
\begin{theorem}{Критерий Рисса}{th:2}
	$\underset{\text{предкомпакт}}{F} \subset L^p (\Omega) \Leftrightarrow$
	\begin{enumerate}
		\item $\exists M \geq 0: \norm{f}_{L^p (\Omega)} \leq M, \forall f \in F$
		\item $\sup\limits_{f\in F}{\sup\limits_{|z| < \rho}{\norm{f(x+z) - f(x)}_{L^p (\Omega)}}} = \delta(\rho) \to 0$
	\end{enumerate}
\end{theorem}

\section{Обобщенные функции}

\subsection{Мотивация}

Пусть $f \in C^1 (\Omega), \varphi \in C_0^\infty (\Omega)$, тогда
\begin{equation*}
	\intset{\Omega}{\frac{\partial f}{\partial x_i} \varphi dx} \overset{\text{по частям}}{=} \intset{\partial \Omega}{f \varphi n_i d\sigma} - \intset{\Omega}{u \frac{\partial \varphi}{\partial x_i} dx} \overset{\varphi \equiv 0 \text{ вне } \Omega}{=} 0 - \intset{\Omega}{u \frac{\partial \varphi}{\partial x_i} dx} = -\intset{\Omega}{u \frac{\partial \varphi}{\partial x_i} dx}
\end{equation*}
или
\begin{equation*}
	\intset{\Omega}{f^\prime \varphi dx} = -\intset{\Omega}{f \varphi^\prime dx}
\end{equation*}
функцию $f^\prime$ удовлетворяющую этим свойствам можно найти для более широкого класса функций.

\subsection{Пространства основных функций}

\begin{definition}{D-сходимость}{def:4}
	Пусть $\Omega \subset \Real^n$ -- открытое, в $C_0^\infty (\Omega)$ определим сходимость (именуемую D-сходимостью) как:
	$\varphi_m \Dto \varphi \Leftrightarrow$
	\begin{enumerate}
		\item $\exists \underset{\text{компакт}}{K} \subset \Omega: \supp \varphi \subset K, \ \supp \varphi_m \subset K$
		\item $D^\alpha \varphi_m \to D^\alpha \varphi \text{ в } C(K), \forall \alpha \text { -- мультииндекса}$
	\end{enumerate}
\end{definition}

\begin{definition}{Пространство основных фунцкий}{def:5}
	$C_0^\infty (\Omega)$ в совокупности с D-сходимостью образует $D(\Omega)$ -- пространство основных функций
\end{definition}

\begin{definition}{Распределение}{def:6}
	Распределением или обобщенной функцией называется линейный функционал $T : D(\Omega) \to \Real$ непрерывный относительно D-сходимости, т.е. такой что:
	\begin{enumerate}
		\item $T(\alpha_1 \varphi_1 + \alpha_2 \varphi_2) = \alpha_1 T(\varphi_1) + \alpha_2 T(\varphi_2), \forall \alpha_1, \alpha_2 \in \Real, \ \forall \varphi_1, \varphi_2 \in D(\Omega)$
		\item $\varphi_m \Dto \varphi \Rightarrow T(\varphi_m) \to T(\varphi)$
	\end{enumerate}
	Множество распределений обозначают как $D^* (\Omega)$
\end{definition}

\begin{proposition}{}{prop:6}
	$D^* (\Omega)$ -- линейное многообразие
\end{proposition}

\begin{example}{Регулярные распределения}{exmpl:2}
	Пусть $f \in L_{\text{loc}}^1 (\Omega)$, определим $T_f (\varphi) = \intset{\Omega}{f \varphi dx}, \forall \varphi \in D(\Omega)$. Можно показать что $T_f \in D^* (\Omega)$, распределения представимые таким образом называют регулярными. Остальные сингулярные.
\end{example}

\begin{example}{$\delta$-функция Дирака}{exmpl:3}
	Считая что $0 \in \Omega$ определим $T_\delta (\varphi) = \varphi(0)$. Можно показать что \\ $T_\delta \in D^* (\Omega)$ и это распределение не является регулярным т.е. сингулярно.
\end{example}

\begin{example}{Ограниченная мера Радона}{exmpl:4}
	Пусть $T \in D^* (\Omega)$ и $\exists M > 0: |T(\varphi)| \leq M \norm{\varphi}_{C(K)}$, где $K$ -- компакт такой что $\supp \varphi \subset K \subset \Omega$. Такое распределение называется ограниченной мерой Радона.
\end{example}

\subsection{Дифференцирование распределений}

\begin{definition}{Производная распределения}{def:7}
	$S = (-1)^{|\alpha|} (T \circ D^\alpha)$ -- это распределение (согласно следующему утверждению) называется частной производной распределения $T$, обозначается как $S = D^\alpha T$
\end{definition}

\begin{proposition}{}{prop:7}
	$D^\alpha T \in D^* (\Omega)$
\end{proposition}
Любое распределение таким образом дифференцируемо любое число раз.

\begin{example}{}{exmpl:5}
	Пусть $\Omega = \Real$ и $\chi(x) = 
	\begin{cases}
		1, &x \geq 0 \\
		0, &x < 0
	\end{cases}$, тогда 
	\begin{multline*}
		\frac{d}{dx}T_{\chi} (\varphi) = -T_{\chi}\left(\frac{d\varphi}{dx}\right) = -\int\limits_{-\infty}^{+\infty}{\chi \frac{d\varphi}{dx} dx} = -\int\limits_{0}^{+\infty}{\frac{d\varphi}{dx} dx} = -\lim_{M\to +\infty}{\int\limits_{0}^{M}{\frac{d\varphi}{dx}}} = \\ =\lim_{M\to +\infty}{\varphi(M) - \varphi(0)} = \varphi(0) - \lim_{M\to +\infty}{\varphi(M)} \overset{\text{финитность } \varphi}{=} \varphi(0) = T_\delta (\varphi)
	\end{multline*}
	т.е. $\frac{d}{dx}T_\chi = T_\delta$
\end{example}

\subsection{Обобщенные производные в смысле Соболева}

\begin{definition}{Обобщенная производная}{def:8}
	Пусть $\underset{\text{открытое}}{\Omega} \subset \Real^n, f \in L_{\text{loc}}^1 (\Omega)$ и $D^\alpha T_f$ -- регулярное распределение т.е. $\exists g \in L_{\text{loc}}^1 (\Omega): D^\alpha T_f (\varphi) = \intset{\Omega}{g \varphi dx}$, тогда $g$ называется обобщенной производной в смысле Соболева от $f$, это обозначается как $g = D_c^\alpha f$
\end{definition}
Нетрудно показать что если $f \in C^{|\alpha|} (\overline{\Omega})$, то $D_c^\alpha f = D^\alpha f$, т.е. это действительно обобщение понятия производной.

\begin{theorem}{О свойствах обобщенных производных}{th:2}
	Будем говорить что $u_m \to u$ в $L_{\text{loc}}^1 (\Omega)$ если $\forall \underset{\text{компакт}}{K} \subset \Omega: u_m \to u \text{ в } L^1 (\Omega)$. 
	\begin{enumerate}
		\item $u \in L_{\text{loc}}^1 (\Omega)$ и $\exists D_c^\alpha u$ в $\Omega$, тогда для $\Omega^\prime \subset \Omega$: $\exists D_c^\alpha (u|_{\Omega^\prime}) \text{ в } \Omega^\prime$ и \\ $D_c^\alpha (u|_{\Omega^\prime}) = (D_c^\alpha u)|_{\Omega^\prime}$
		\item $u_1, u_2 \in L_{\text{loc}}^1 (\Omega), c_1, c_2 \in \Real$ и $\exists D_c^\alpha u_1, D_c^\alpha u_2$ в $\Omega$, тогда \\ $\exists D_c^\alpha (c_1 u_1 + c_2 u_2) = c_1 D_c^\alpha u_1 + c_2 D_c^\alpha u_2$
		\item $\begin{cases} 
				D_c^\alpha u_m \to v &\text{ в } L_{\text{loc}}^1 (\Omega) \\
				u_m \to u &\text{ в } L_{\text{loc}}^1 (\Omega) 
				\end{cases}$, тогда $\exists D_c^\alpha u = v$
	\end{enumerate}
	Свойство 3 называется признаком обобщенной производной
\end{theorem}

\subsection{Усреднение функций имеющих обобщенные производные}

\begin{lemma}{}{}
	Пусть $u \in L^1 (\Omega)$ и $\exists D_c^\alpha u \in L_{\text{loc}}^1 (\Omega)$, $x\in \Omega: 0 < \rho < \dist(x, \partial \Omega)$, тогда $(D_c^\alpha u)_\rho (x) = (D^\alpha u_\rho) (x)$
\end{lemma}
Условие на $\rho$ существенно, в этом случае $\overline{B}(x, \rho) \subset \Omega$ т.е. $\omega_\rho (x - y)$ финитна по $y$ в $\Omega$.

\begin{lemma}{}{}
	Пусть $u \in L^p (\Omega), 1 \leq p < \infty$ и $\exists D_c^\alpha \in L^p (\Omega)$, тогда $\forall \Omega^\prime \ssubset \Omega: 
	\begin{cases} 
		u_\rho \to u &\text{ в } L^p (\Omega^\prime) \\ 
		D^\alpha (u_\rho) \to D_c^\alpha u &\text{ в } L^p (\Omega^\prime)
	\end{cases}$
\end{lemma}

\begin{lemma}{}{}
	Пусть $u \in L_{\text{loc}}^1 (\Omega)$ и $D_c^\alpha = 0 \text{ в } \Omega, \forall \alpha: |\alpha| = 1$ (все обобщенные производные 1-го порядка равны 0), тогда $u = const$ п.в. в $\Omega$
\end{lemma}

\begin{theorem}{Обобщенная производная в новых координатах}{th:3}
	Пусть $\Omega^\prime, \Omega \subset \Real^n, \varphi: \Omega^\prime \to \Omega$ -- диффеоморфизм, $u \in L_{\text{loc}}^1 (\Omega)$ и $\exists \frac{\partial_c u}{\partial x_i} \in L_{\text{loc}}^1 (\Omega), \forall i \in {1, ..., n}$, тогда $v = u \circ \varphi \in L_{\text{loc}}^1 (\Omega^\prime)$ и
	\begin{enumerate}
		\item $\exists \frac{\partial_c v}{\partial y_i} \in L_{\text{loc}}^1 (\Omega^\prime)$
		\item $\frac{\partial_c v}{\partial y_i} = \sum\limits_{k=1}^{n}{(\frac{\partial_c u}{\partial x_k} \circ \varphi)\frac{\partial \varphi}{\partial y_i}}$
	\end{enumerate}
\end{theorem}

\subsection{Случай одной независимой переменной}

\begin{definition}{Абсолютная непрерывность}{def:9}
	Функция $u: [a, b] \to \Real$ называется абсолютно непрерывной если \\ $\exists v \in L^1 ([a, b]): u(x) = \int\limits_a^x{v(t)dt} + u(a)$
\end{definition}
Можно доказать что абсолютно непрерывная функция равномерно непрерывна.

\begin{theorem}{Критерий существования обобщенной производной}{th:4}
	Для того чтобы $u \in L_{\text{loc}}^1 ([a, b])$ имела 1-ую обобщенную производную $\frac{d_c u}{dx} \in L^1 ([a, b])$ необходимо и достаточно чтобы она была абсолютно непрерывной.
\end{theorem}
\textbf{Мораль:} если $u$ -- суммируема и имеет суммируемую обобщенную производную, то она непрерывна.

\subsection{Пространства Соболева}

\begin{definition}{Пространства $W_p^\ell$}{def:10}
	Пусть $\underset{\text{открыт. огр.}}{\Omega} \subset \Real^n, 1 \leq p \leq \infty, \ell = 0, 1, 2, ...$, тогда определим пространства Соболева
	\begin{equation*}
	 	W_p^\ell (\Omega) = \{f \in L^p (\Omega) \ | \ \exists D_c^\alpha f \in L^p (\Omega), \forall \alpha: |\alpha| \leq \ell\}
	\end{equation*}
	Введем норму
	\begin{equation*}
		\norm{u}_{W_p^\ell (\Omega)} = \norm{u}_{p, \ell, \Omega} = \left( \intset{\Omega}{\sum\limits_{|\alpha| \leq \ell}{|D_c^\alpha u|^p}} \right)^{\frac{1}{p}}
	\end{equation*}
\end{definition}
Можно ввести и другую норму $\nnorm{u}_{p, \ell, \Omega} = \sum\limits_{|\alpha| \leq \ell}{\norm{D_c^\alpha u}_{L^p (\Omega)}}$

\begin{theorem}{}{th:5}
	$\left(W_p^\ell (\Omega), \norm{\cdot}_{p, \ell, \Omega}\right)$ -- банахово
\end{theorem}

\begin{proposition}{}{prop:7}
	$\norm{u}_{p, \ell, \Omega} \sim \nnorm{u}_{p, \ell, \Omega}$
\end{proposition}

\begin{definition}{Пространства $[L^p (\Omega)]^N$}{def:11}
	Определим пространства вектор-функций Лебега
	\begin{equation*}
		[L^p (\Omega)]^N = \{(u_1, ..., u_N) = \underline{u} \ | \ u_1, ..., u_N \in L^p (\Omega)\} 
	\end{equation*}
	Введем норму
	\begin{equation*}
		\norm{\underline{u}}_{[L^p (\Omega)]^N} = \left(\sum\limits_{i=1}^{N}{\intset{\Omega}{|u_i|^p dx}}\right)^{\frac{1}{p}}
	\end{equation*}
\end{definition}
Зачем вводятся эти пространства? На самом деле между $W_p^\ell (\Omega)$ и некоторым подпространством $[L^p (\Omega)]^N$, где $N = \sum\limits_{|\alpha| \leq \ell}{1}$ существует изометрический изоморфизм определяемый как $p: W_p^\ell (\Omega) \to [L^p (\Omega)]^N, p(u) = (D_c^\alpha u)_{|\alpha| \leq \ell}$ при лексикографическом упорядочивании мультииндексов. Полезно изучить их свойства.

\begin{theorem}{Теорема Рисса для $[L^p (\Omega)]^N$}{th:6}
	Пусть $1 < p < \infty$, тогда
	\begin{enumerate}
		\item $\left(\underline{v} \in [L^{p^*} (\Omega)]^N\right) \wedge \left(f(\underline{u}) = \sum\limits_{i=1}^N{\intset{\Omega}{u_i v_i dx}}\right) \Rightarrow \left(f \in ([L^p (\Omega)]^N)^*\right) \wedge \\ \wedge \left(\norm{f}_{([L^p (\Omega)]^N)^*} = \norm{\underline{v}}_{[L^{p^*} (\Omega)]^N}\right)$
		\item $\left(f \in ([L^p (\Omega)]^N)^*\right) \Rightarrow \exists! \underline{v} \in  [L^{p^*} (\Omega)]^N: \left(f(\underline{u}) = \sum\limits_{i=1}^N{\intset{\Omega}{u_i v_i dx}}\right) \wedge \\ \wedge \left(\norm{f}_{([L^p (\Omega)]^N)^*} = \norm{\underline{v}}_{[L^{p^*} (\Omega)]^N}\right)$
	\end{enumerate}
\end{theorem}
\textbf{Мораль:} $[L^p (\Omega)]^N$ рефлексивны, также можно показать что эти пространства сепарабельны, а значит с учетом рефлексивности в них выполнен принцип выбора.

\begin{proposition}{}{prop:8}
	$W_p^\ell (\Omega)$ рефлексивно при $1 < p < \infty$
\end{proposition}
Они также сепарабельны, а значит в них выполнен принцип выбора.

\begin{definition}{Диффеоморфизм класса $C^\ell$}{def:12}
	$\varphi: \Omega \to \omega$ -- диффеоморфизм класса $C^\ell, \ell \in \Natural$, если $\varphi$ -- биекция, $\det \varphi^\prime \neq 0$ и $\varphi$ имеет непрерывные производные до порядка $\ell$ включительно.
\end{definition}

\begin{proposition}{}{prop:9}
	Пусть $u \in W_p^\ell (\Omega)$, $\varphi: \Omega \to \omega$ -- диффеоморфизм класса $C^\ell$ и \\ $v(y) = u(\varphi^{-1} (y)), \forall y \in \omega$, тогда
	\begin{enumerate}
		\item $v \in W_p^\ell (\omega)$
		\item $\exists c_1, c_2 > 0: c_1 \norm{u}_{p, \ell, \Omega} \leq \norm{v}_{p, \ell, \omega} \leq c_2 \norm{u}_{p, \ell, \Omega}$
	\end{enumerate}
\end{proposition}

\subsection{Пространства $\mathring{W_p^\ell}$}

\begin{definition}{Пространства $\mathring{W_p^\ell}$}{def:13}
	Определим т.н. пространства Соболева с нулевыми граничными условиями:
	\begin{equation*}
		\mathring{W_p^\ell} (\Omega) = \overline{C_0^\infty (\Omega)} \text{ в } W_p^\ell
	\end{equation*}
\end{definition}
Это замыкание линейного многообразия т.е. подпространство в $W_p^\ell (\Omega)$. Т.к. $W_p^\ell (\Omega)$ при $1 \leq p < \infty$ сепарабельны и при $1 < p < \infty$ рефлексивны, то для $\mathring{W_p^\ell} (\Omega)$ эти свойства сохраняются.

\begin{proposition}{}{prop:10}
	Пусть $u \in \mathring{W_p^\ell} (\Omega), \Omega \subset \tilde{\Omega}$ и пусть
	$\tilde{u} (x) = 
		\begin{cases}
			u(x), &x \in \Omega \\
			0, &x \in \tilde{\Omega} \setminus \Omega
	 	\end{cases}$ п.в., тогда $\tilde{u} \in \mathring{W_p^\ell} (\tilde{\Omega})$ и $\norm{\tilde{u}}_{p, \ell, \tilde{\Omega}} = \norm{u}_{p, \ell, \Omega}$
\end{proposition}

\begin{example}{}{exmpl:6}
	Пусть $n = 1, \ell = 1, p = 1$ и $\Omega = (0, 1)$, зададим $u \equiv 1$, тогда $u \in W_1^1 (\Omega)$. Допустим $u \in \mathring{W_1^1}$, $\tilde{\Omega} = (0, 2)$ и 
	$\tilde{u} = 
		\begin{cases}
			1, &x \in (0, 1) \\
			0, &x \in [1, 2)
		\end{cases}$, тогда \\ $\tilde{u} \in \mathring{W_1^1} (\tilde{\Omega}) \subset W_1^1 (\tilde{\Omega})$, но $\tilde{u}$ разрывна на $\tilde{\Omega} = (0, 2)$, а так не может быть согласно критерию существования 1-ой обобщенной производной.
\end{example}
\textbf{Мораль:} $\mathring{W_p^\ell} (\Omega) \neq W_p^\ell (\Omega)$

Рассмотрим несколько важных свойств этих пространств.
\begin{proposition}{}{prop:11}
	$u \in \mathring{W_p^\ell} (\Omega)$, тогда $u_\rho \to u$ в $W_p^\ell (\Omega)$
\end{proposition}

\begin{proposition}{}{prop:12}
	Пусть $u \in W_p^\ell (\Omega)$ и $\supp{u} \ssubset \Omega$, тогда $u \in \mathring{W_p^\ell}$
\end{proposition}

\begin{proposition}{Интегрирование по частям}{prop:13}
	\begin{enumerate}
		\item Пусть $u \in W_p^1 (\Omega)$, $v \in \mathring{W_{p^*}^1} (\Omega)$, тогда
			\begin{equation*}
			\intset{\Omega}{u \frac{\partial_c v}{\partial x_i} dx} = -\intset{\Omega}{\frac{\partial_c u}{\partial x_i} v dx}
			\end{equation*}
		\item Пусть $u \in W_p^\ell (\Omega)$, $v \in \mathring{W_{p^*}^\ell} (\Omega)$, тогда
			\begin{equation*}
			\intset{\Omega}{u D_c^\alpha v dx} = -\intset{\Omega}{v D_c^\alpha u dx}
			\end{equation*}
	\end{enumerate}
	2-ая часть утверждения прямое следствие 1-ой (при этом её обобщение)
\end{proposition}

\begin{theorem}{Неравенство Фридрихса}{th:7}
	Для $u \in \mathring{W_p^\ell} (\Omega)$ определим 
	\begin{equation*}
		|u|_{p, \ell, \Omega} = \left(\intset{\Omega}{\sum_{|\alpha| = \ell}{|D_c^\alpha u|^p} dx}\right)^{\frac{1}{p}}
	\end{equation*}
	это полунорма. Пусть $d = \diam{\Omega}$, тогда
	\begin{equation*}
		\norm{u}_{p, \ell, \Omega} \leq d^\ell |u|_{p, \ell, \Omega}, \forall u \in \mathring{W_p^\ell} (\Omega)
	\end{equation*}
\end{theorem}
\textbf{Мораль:} в $\mathring{W_p^\ell} (\Omega)$: $|\cdot|_{p, \ell, \Omega}$ норма эквивалентная $\norm{\cdot}_{p, \ell, \Omega}$.

\begin{example}{Интересное скалярное произведение}{exmpl:7}
	В $\mathring{W_2^2} (\Omega)$ можно ввести следующее скалярное произведение:
	\begin{equation*}
		(u, v)_{\mathring{W_2^2} (\Omega)} = \intset{\Omega}{\nabla^2 u \cdot \nabla^2 v dx}
	\end{equation*}
	где подразумевается $\nabla^2 = \sum\limits_{|\alpha| = 2}{D_c^\alpha}$ (это не лапласиан!)
\end{example}

\subsection{Двойственность в пространствах Соболева. Пространства Соболева с отрицательными индексами}

\begin{theorem}{}{th:8}
	Пусть $f \in \left(W_p^\ell (\Omega)\right)^*$, тогда $\exists v = (v_\alpha)_{|\alpha| \leq \ell} \in [L^{p^*} (\Omega)]^N$:
	\begin{align*}
		&f(u) = \intset{\Omega}{\sum\limits_{|\alpha| \leq \ell}{(D_c^\alpha u \cdot v_\alpha)} dx}, \forall u \in W_p^\ell (\Omega) \\
		&\norm{f}_{\left(W_p^\ell (\Omega)\right)^*} = \norm{v}_{[L^{p^*} (\Omega)]^N}
	\end{align*}
\end{theorem}

\begin{definition}{Пространства $W_p^{-\ell} (\Omega)$}{def:14}
	Т.н. пространства Соболева с отрицательными индексами составляют обобщенные функции:
	\begin{equation*}
		W_p^{-\ell} (\Omega) = \{ T \in D^* (\Omega) \ | \ T = \sum\limits_{|\alpha| \leq \ell}{(-1)^{|\alpha|} D^\alpha T_{v_\alpha}}, \text { где } v_\alpha \in L^p (\Omega), \forall |\alpha| \leq \ell \}
	\end{equation*}
\end{definition}
В сущности $W_p^\ell (\Omega)$ состоит из первообразных функций из $L^p$ (т.к. существуют обобщенные производные), а $W_p^{-\ell} (\Omega)$ уже состоит из производных (в виде обобщенных функций).

\begin{theorem}{}{th:9}
	$\left(\mathring{W_{p^*}^\ell} (\Omega)\right)^*$ изометрически изоморфно $W_p^{-\ell} (\Omega)$
\end{theorem}
\textbf{Мораль:} $W_p^{-\ell} (\Omega)$ -- банахово пространство, сепарабельное и рефлексивное. 

\begin{example}{Где живет $\delta$-функция?}{exmpl:8}
	Возьмем $v = (0, -\chi, 0, ...)$, всего $\ell + 1$ элементов (работаем в $\Real$). Получаем 
	\begin{equation*}
		T_v (\varphi) = \sum\limits_{|\alpha| \leq \ell}{(-1)^{|\alpha|} D^\alpha T_{v_\alpha}(\varphi)} = (-1)D^1 T_{-\chi} (\varphi) = (-1)^2 D^1 T_{\chi} = D^1 T_{\chi} = T_\delta (\varphi)
	\end{equation*}
	и поскольку $-\chi \in L^p (\Omega)$, то $T_\delta \in W_p^{\ell}, \forall \ell \in \Natural, 1 \leq p \leq \infty$
\end{example}

\newpage

\tcblistof[\section*]{def}{Список определений}

\end{document}
