\documentclass[12pt,a4paper]{article}
\usepackage[utf8]{inputenc}
\usepackage[english,russian]{babel}
\usepackage{hyperref}
\hypersetup{
	colorlinks   = true, 
	urlcolor     = blue, 
	linkcolor    = blue, 
	citecolor   = red
}
\usepackage{amsthm,amssymb,amsfonts,amsmath}
\usepackage{physics}
\usepackage{xcolor}
\usepackage[left=3cm,right=3cm,
top=3cm,bottom=3cm,bindingoffset=0cm]{geometry}
\usepackage{tcolorbox}
\usepackage{physics}

\tcbuselibrary{theorems}
\newtcbtheorem
[number within=section]
{definition}
{Определение}
{
	colback=red!3,
	colframe=red!20!white,
	coltitle=black,
	fonttitle=\bfseries,
	sharp corners,
}
{def}

\tcbuselibrary{theorems}
\newtcbtheorem
[number within=section]
{proposition}
{Утверждение}
{
	colback=purple!3,
	colframe=purple!25!white,
	coltitle=black,
	fonttitle=\bfseries,
}
{prop}

\tcbuselibrary{theorems}
\newtcbtheorem
[number within=section]
{theorem}
{Теорема}
{
	colback=blue!3,
	colframe=blue!20!white,
	coltitle=black,
	fonttitle=\bfseries,
}
{th}

\tcbuselibrary{theorems}
\newtcbtheorem
[number within=section]
{example}
{Пример}
{
	colback=orange!3,
	colframe=orange!20!white,
	coltitle=black,
	fonttitle=\bfseries,
	sharp corners,
}
{exmpl}

\tcbuselibrary{theorems}
\newtcbtheorem
[number within=section]
{lemma}
{Лемма}
{
	colback=gray!3,
	colframe=gray!20!white,
	coltitle=black,
	fonttitle=\bfseries,
}
{lemm}

% Для удобства
\newcommand{\intset}[1]{\int\limits_{#1}}
\newcommand{\Real}{\mathbb{R}}
\newcommand{\ssubset}{\subset \subset}

\title{Соболевские пространства}

\begin{document}
	
\maketitle
\begin{abstract}
\end{abstract}

\section{Усреднение функций}
	
\subsection{Основные понятия}

Пусть в $\Real^n$ задана норма 
$$\norm{x} = \sqrt{\sum_{i=1}^{n}{x_i^2}}$$ 
Определим $\omega: \Real^n \to \Real$ как
\begin{equation*}
	\omega (x) = 
		\begin{cases}
		c e^{\frac{1}{\norm{x}^2 - 1}}, & x \in B(0, 1) \\
		0, & x \notin B(0, 1)
		\end{cases}
\end{equation*}
где 
$$c  = \frac{1}{\intset{B(0,1)}{e^{\frac{1}{\norm{x}^2 - 1}}dx}}$$
так что $\intset{B(0,1)}{\omega(x)dx} = 1$ и кроме того по определению $\omega \in C(\Real^n)$. \\ Пусть $\omega_\rho (x) = \rho^{-n} \omega \left(\frac{x}{\rho}\right)$, тогда свойства указанные выше сохраняются.

\begin{definition}{Усреднение функции}{def:1}
	Пусть $\rho > 0$, усреднением функции $u \in L^{1}(\Omega)$ называется 
	$$u_\rho (x) = \intset{\Omega}{\omega_\rho (x - y) u(y) dy}, \forall x \in \Real^n$$
	Важное свойство $u_\rho \in C^{\infty}(\Real^n), \forall \rho > 0$
\end{definition}

\begin{proposition}{Вложенность $L^p$}{prop:1}
	Пусть $\Omega \subset \Real^n$ -- открытое ограниченное, тогда
	\begin{enumerate}
		\item $L^{\infty}(\Omega) \subset \underset{1 < p < \infty}{L^{p}(\Omega)} \subset L^{1}(\Omega)$
		\item $L^{p}(\Omega) \subset L^{q}(\Omega), 1 \leq q < p \leq \infty$
	\end{enumerate}
\end{proposition}

\begin{proposition}{О норме усреднения}{prop:2}
	Пусть $u \in L^{p}(\Omega), 1 \leq p \leq \infty$, тогда 
	$$\norm{u_\rho}_{L^{p}(\Omega)} \leq \norm{u}_{L^{p}(\Omega)}$$
\end{proposition}

\begin{proposition}{О сходимости усреднений в $C$}{prop:3}
	Пусть $u \in C(\Omega)$, $\underset{\text{компакт}}{K} \subset \Omega$, тогда $u_\rho \underset{\rho \to 0}{\to} u$ в $C(K)$
\end{proposition}
Из последнего утверждения следует что $u_\rho (x) \underset{\rho \to 0}{\to} u (x), \forall x \in \Omega$.

\begin{theorem}{О сходимости усреднений в $L^{p}$}{th:1}
	Пусть $u \in L^{p}(\Omega), 1 \leq p < \infty$, тогда $u_\rho \underset{\rho \to 0}{\to} u$ в $L^{p} (\Omega)$
\end{theorem}

\subsection{Свойства функций из $L^p$ связанные с усреднением}

Рассмотрим множества плотные в пространствах Лебега.
\begin{proposition}{}{prop:4}
	$C^{\infty} (\overline{\Omega})$ плотно в $L^p (\Omega), 1 \leq p < \infty$
\end{proposition}

\begin{proposition}{}{prop:5}
	$C^{\infty}_0 (\Omega)$ плотно в $L^p (\Omega), 1 \leq p < \infty$
\end{proposition}

\begin{definition}{Вхождение с замыканием}{def:2}
	$\Omega^{'} \ssubset \Omega$ если $\Omega^{'}$ -- предкомпакт и $\overline{\Omega^{'}} \subset \Omega$
\end{definition}

\begin{definition}{Локально суммируемые функции}{def:3}
	$u \in L^{p}_{\text{loc}} (\Omega)$ -- $u$ локально суммируема с показателем $p$, если \\ $u \in L^{p} (\Omega^{'}), \forall \Omega^{'} \ssubset \Omega$
\end{definition}
Заметим что на ограниченной области локально суммируемые функции ведут себя как угодно на границе.

\begin{example}{}{exmpl:1}
	$f \equiv c$, $f \notin L^1 (\Real)$, но $f \in L^1_{\text{loc}} (\Real)$
\end{example}

\begin{lemma}{Дюбуа-Реймонда}{}
	Пусть $u \in L^1_{\text{loc}} (\Omega)$ и $\intset{\Omega}{u\phi dx}, \forall \phi \in C^{\infty}_0 (\Omega)$, тогда $u = 0$ п.в. на $\Omega$ [Заметим что класс для $u$ очень широкий, а для $\phi$ очень узкий]
\end{lemma}

\subsection{Критерий компактности в $L^p$}

Пусть $F \subset L^p (\Omega), 1 \leq p < \infty$, по умолчанию $f \in F$ продолжаем 0 на $\Real^n$
\begin{theorem}{Критерий Рисса}{th:2}
	$\underset{\text{предкомпакт}}{F} \subset L^p (\Omega) \Leftrightarrow$
	\begin{enumerate}
		\item $\exists M \geq 0: \norm{f}_{L^p (\Omega)} \leq M, \forall f \in F$
		\item $\sup\limits_{f\in F}{\sup\limits_{|z| < \rho}{\norm{f(x+z) - f(z)}_{L^p (\Omega)}}} = \delta(\rho) \to 0$
	\end{enumerate}
\end{theorem}

\end{document}
